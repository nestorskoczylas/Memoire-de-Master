% Fiche nº0

\chapter{Synthèse et positionnement des articles} % Main appendix title
\label{app:Fiche0} % For referencing this appendix elsewhere, use \ref{app:Fiche0}

%%%%%%%%%%%%%%%%%%%%%%%%%%%%%%%%%%%%%%%%%%%%%%%%%%%%%%%%%%%%%%%%%%%%%%%%%%%%%%%%%%
\section{Description de la problématique}

\textbf{Problématique choisie :} \emph{Comment incorporer les principes de développement durable et écologique dans le processus de développement de logiciels en vue de créer des applications plus respectueuses de l'environnement ?}

\vspace{0.5cm}

Cette problématique est intéressante, car elle explore un domaine peu connu et peu étudié par la communauté scientifique, le développement logiciel vert. De plus, elle se concentre sur la possibilité d'intégrer une perspective écologique au développement de logiciels, tout en soulignant l'importance de la durabilité environnementale.

\singlespacing
\noindent L'objectif principal est de comprendre comment le développement logiciel peut être conçus et développé de manière à prendre en compte les aspects écologiques, tels que la réduction de la consommation d'énergie, l'optimisation des ressources et la minimisation de l'impact environnemental, tout au long du cycle de vie du logiciel.
Au coeur de cette problématique se trouve l'idée de proposer des méthodes et des pratiques concrètes pour intégrer la durabilité et le développement durable dans les processus de développement logiciel et donc favoriser la création de logiciels écologiques et plus respectueux de l'environnement.

\singlespacing
\noindent Cette problématique reste d'une grande actualité en lien avec les préoccupations environnementales contemporaines. Elle souligne l'importance croissante de la durabilité environnementale dans le domaine du développement logiciel et propose des solutions concrètes pour contribuer à la création d'applications informatiques plus respectueuses de l'environnement.

\section{Synthèse des articles}
Cette synthèse comparative présente les articles principaux et secondaires que vous avez fournis, en se concentrant sur le domaine de l'Ingénierie Logicielle Verte. L'objectif est de fournir une vue d'ensemble de la recherche actuelle sur ce sujet.

\section{Tableaux comparatifs}

On peut synthétiser les articles selon deux axes principaux :

\begin{itemize}
    \item Articles principaux et articles secondaires
    \item Objectif de l'article (comprendre, modéliser, intégrer la durabilité dans la pratique)
\end{itemize}

\subsection{Articles principaux et articles secondaires}

\begin{table}[htb]
    \small
    \renewcommand{\arraystretch}{1.5}
    \begin{tabularx}{\textwidth}{|X|c|c|}
        \hline
        \textbf{Article} & \textbf{Principal/Secondaire} & \textbf{Fiche}\\ 
        \hline
        Green Software Engineering with Agile Methods & Principal & \ref{app:Fiche1} \\ \hline
        The GREENSOFT Model: A reference model for green and sustainable software and its engineering & Principal & \ref{app:Fiche2} \\ \hline
        An Empirical Study of Practitioners’ Perspectives on Green Software Engineering & Principal & \ref{app:Fiche3} \\ \hline
        Safety, Security, Now Sustainability: The Nonfunctional Requirement for the 21st Century & Principal & \ref{app:Fiche4} \\ \hline
        Integrating Sustainability Metrics into Project and Portfolio Performance Assessment in Agile Software Development: A Data-Driven Scoring Model & Secondaire & \ref{app:Fiche5} \\ \hline
        Green and Sustainable Software Engineering - a Systematic Mapping Study & Secondaire & \ref{app:Fiche6} \\ \hline
        Sustainability is Stratified: Toward a Better Theory of Sustainable Software Engineering & Secondaire & \ref{app:Fiche7} \\ \hline
        The Green Software Measurement Structure Based on Sustainability Perspective & Secondaire & \ref{app:Fiche8} \\ \hline
        Sustainable software engineering – have we neglected the software engineer’s perspective? & Secondaire & \ref{app:Fiche9} \\ \hline
        The Impact of Green Feedback on Users’ Software Usage & Secondaire & \ref{app:Fiche10} \\
        \hline
        Application of the Sustainability Awareness Framework in Agile Software Development & Secondaire & \ref{app:Fiche11} \\ \hline
        From Sustainability in Requirements Engineering to a Sustainability-Aware Scrum Framework & Secondaire & \ref{app:Fiche12} \\ \hline
    \end{tabularx}
    \caption{Articles principaux et articles secondaires}
\end{table}

\subsection{Objectif de l'article}

\begin{table}[htb]
    \small
    \renewcommand{\arraystretch}{1.5}
    \begin{tabularx}{\textwidth}{|X|X|}
        \hline
        \textbf{Article} & \textbf{Objectif} \\ 
        \hline
        Green Software Engineering with Agile Methods & Proposer un modèle d'intégration du développement durable dans les méthodes agiles \\ \hline
        The GREENSOFT Model: A reference model for green and sustainable software and its engineering & Présenter un modèle de référence pour les logiciels durables \\ \hline
        An Empirical Study of Practitioners’ Perspectives on Green Software Engineering & Étudier la perception du développement durable par les ingénieurs logiciels \\ \hline
        Safety, Security, Now Sustainability: The Nonfunctional Requirement for the 21st Century & Argumenter pour la prise en compte du développement durable comme exigence non fonctionnelle \\ \hline
        Integrating Sustainability Metrics into Project and Portfolio Performance Assessment in Agile Software Development: A Data-Driven Scoring Model & Proposer un modèle d'évaluation intégrant la durabilité dans la gestion de projet agile \\ \hline
        Green and Sustainable Software Engineering - a Systematic Mapping Study & Analyser l'état de l'art du génie logiciel durable \\ \hline
        Sustainability is Stratified: Toward a Better Theory of Sustainable Software Engineering & Proposer une meilleure théorie du génie logiciel durable \\ \hline
        The Green Software Measurement Structure Based on Sustainability Perspective & Définir une structure de mesure du logiciel durable \\ \hline
        Sustainable software engineering – have we neglected the software engineer’s perspective? & Souligner l'importance de la dimension individuelle du développement durable \\ \hline
        The Impact of Green Feedback on Users’ Software Usage & Étudier l'impact d'un retour d'information écologique sur l'utilisation des logiciels par les utilisateurs finaux \\
        \hline
        Application of the Sustainability Awareness Framework in Agile Software Development & Evaluer l'efficacité d'un framework pour intégrer le développement durable dans les processus agiles \\ \hline
        From Sustainability in Requirements Engineering to a Sustainability-Aware Scrum Framework & Adapter les méthodes agiles pour intégrer le développement durable \\ \hline
    \end{tabularx}
    \caption{Objectif de l'article}
\end{table}

Le domaine du développement durable et écologique dans le génie logiciel est en plein essor. Il existe un intérêt croissant pour la création de logiciels plus respectueux de l'environnement, et les chercheurs explorent différentes approches pour intégrer les principes de durabilité dans les processus de développement logiciel.

\singlespacing
\noindent Les articles analysés dans ce document présentent un éventail de perspectives et de contributions à ce domaine. Ils soulignent l'importance de considérer l'impact environnemental du logiciel tout au long de son cycle de vie, et proposent des solutions concrètes pour réduire la consommation d'énergie, optimiser les ressources et minimiser l'empreinte carbone des logiciels.
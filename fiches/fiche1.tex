% Fiche nº1

\chapter{Fiche nº1} % Main appendix title
\label{app:Fiche1} % For referencing this appendix elsewhere, use \ref{app:Fiche1}

%%%%%%%%%%%%%%%%%%%%%%%%%%%%%%%%%%%%%%%%%%%%%%%%%%%%%%%%%%%%%%%%%%%%%%%%%%%%%%%%%%
\section{Description de l'article}

\paragraph{Titre de l'article~: \textnormal{Green Software Engineering with Agile Methods}}
\paragraph{Lien de l'article~: \textnormal{https://ieeexplore.ieee.org/document/6606425}}
\paragraph{Liste des auteurs~: \textnormal{Markus Dick, Jakob Drangmeister, Eva Kern, Stefan Naumann}}
\paragraph{Affiliation des auteurs~: \textnormal{Trier University Of Applied Sciences, Environmental Campus Birkenfeld, Germany}}
\paragraph{Nom de la conférence / revue~: \textnormal{ICSE International Conference on Software Engineering}}
\paragraph{Classification de la conférence / revue~: \textnormal{A*}}
\paragraph{Nombre de citations de l'article (quelle source ?)~: \textnormal{53 citations (Semantic Scholar)}}



%%%%%%%%%%%%%%%%%%%%%%%%%%%%%%%%%%%%%%%%%%%%%%%%%%%%%%%%%%%%%%%%%%%%%%%%%%%%%%%%%%
\section{Synthèse de l'article}

\paragraph{Problématique}
L’article “Green Software Engineering with Agile Methods” expose plusieurs problématiques. Dans un premier temps, l'article met en évidence le manque d’attention associé à la durabilité dans le développement et génie logiciel. Peu d’efforts ont été dédiés au développement de logiciels durable, en dépit de la consommation croissante d’énergie des technologies de l’information et de la communication (TIC) ainsi que son impact sur le changement climatique. Cette carence est préoccupante étant donné que les logiciels représentent une part significative de la consommation énergétique globale. De ce fait, il devient essentiel de développer des pratiques de développement de logiciels durables pour atténuer leur impact environnemental.

Dans un deuxième temps, l’article fait ressortir le besoin de recherches supplémentaires pour déterminer la façon de développer des logiciels durables. Même si le modèle proposé inclut les aspects de durabilité dans la méthodologie Scrum, il n’offre pas de preuves justifiant que ces mesures amènent à des produits étant plus efficaces et ayant une empreinte carbone réduit. C’est pourquoi, il est désormais nécessaire d’approfondir les recherches pour déterminer si l’intégration des aspects de durabilité dans Scrum peut être à même d’aider à créer des logiciels dit “plus vert” au moment de leur conception. Cela nécessite l'élaboration de critères et des métriques appropriés pour évaluer l'efficacité des pratiques de développement de logiciels durables.

Enfin, dans un troisième temps, l’article met en lumière l’importance de mesurer et de surveiller ces aspects de durabilité dans le développement de logiciels. Privé de ces pratiques, il est impossible de déterminer la performance d’un produit logiciel en termes de durabilité. Dès lors, il devient crucial aux développeurs de prendre conscience des enjeux et de leur impact sur la consommation d’énergie, pour enfin prendre des mesures pour les réduire. Cela nécessite une sensibilisation et une intégration systématique de la durabilité dans les processus de développement de logiciels.

\paragraph{Pistes possibles (pointés par les auteurs)}
Afin de remédier à ces problématiques, les auteurs proposent d’intégrer des améliorations d’ingénierie logicielle durable et verte dans la méthodologie Scrum. Ce qui implique de repenser les pratiques de développement, d’établir des critères précis et de mettre en place des mécanismes de mesures et de surveillance adéquate. Les développeurs doivent être sensibilisés à l'importance de la durabilité et être encouragés à adopter des approches plus écologiques dans leur travail.

\paragraph{Questions de recherche}
\begin{itemize}
    \item Quelles mesures peuvent être prises pour favoriser le développement de logiciels durable, intégrant des aspects de durabilité, tout en réduisant la consommation d'énergie et l'empreinte carbone ?
    \item Comment pouvons-nous évaluer l'efficacité de ces mesures et créer des métriques appropriées pour mesurer la durabilité dans le développement de logiciels ?
    \item Quelles sont les meilleures pratiques à adopter pour intégrer la durabilité dans les processus de développement de logiciels existants ?
    \item Quelles sont les implications de l'intégration de l'ingénierie logicielle durable dans la méthodologie agile Scrum ?
\end{itemize}

\paragraph{Démarche adoptée}
Les auteurs soumettent ce modèle d'intégration des améliorations de l'ingénierie logicielle verte et durable dans Scrum, une méthodologie agile pour le développement de logiciels. Ils suggèrent plusieurs voies possibles pour la mise en œuvre de ce modèle :
\begin{enumerate}
    \item Définir des objectifs de durabilité : l'élaboration du modèle proposé consiste à définir des objectifs de durabilité pour le produit logiciel. Il s'agit d'identifier l'impact environnemental du produit et de fixer des objectifs de réduction de cet impact.
    \item Intégrer les aspects de durabilité dans Scrum : ils proposent plusieurs façons d'intégrer les aspects de durabilité dans Scrum, notamment en ajoutant de nouveaux rôles et responsabilités à l'équipe Scrum, en incorporant des critères de durabilité dans les récits des utilisateurs et en utilisant des mesures de l'énergie dans l'intégration continue.
    \item Mesurer et contrôler les aspects de durabilité : pour s'assurer que les objectifs de durabilité soient atteints, il faut mesurer et contrôler les paramètres pertinents tout au long du processus de développement du logiciel. Ils suggèrent d'utiliser un ensemble de mesures développées par Albertao \cite{Albertao} pour évaluer les questions de durabilité dans les projets logiciels réels.
    \item Améliorer continuellement les projets logiciels : en mesurant un ensemble de paramètres de manières répétées au cours de plusieurs itérations, il est possible d'améliorer à chaque instant les projets de logiciels concernés par les questions de durabilité.
\end{enumerate}

\paragraph{Implémentation de la démarche}
L'implémentation de l'approche proposée dans cet article comporte plusieurs étapes :
\begin{enumerate}
    \item Établissement des objectifs de durabilité : la première étape consiste à définir les objectifs de durabilité du produit logiciel en identifiant son impact environnemental et en fixant des objectifs visant à réduire cet impact.
    \item Identification des parties prenantes pertinentes : la prochaine étape consiste à identifier les parties prenantes qui jouent un rôle important dans le processus de développement logiciel et qui sont concernées par les questions de durabilité.
    \item Intégration de critères de durabilité dans les user stories : les auteurs suggèrent d'intégrer des critères de durabilité dans les user stories, qui sont de courtes descriptions des fonctionnalités attendues d'un produit logiciel. En incluant des critères de durabilité dans les user stories, les développeurs peuvent s'assurer que les considérations environnementales sont prises en compte tout au long du processus de développement.
    \item Attribution de nouveaux rôles et responsabilités à l'équipe Scrum : les auteurs proposent d'ajouter de nouveaux rôles et responsabilités à l'équipe Scrum, tels qu'un responsable du développement durable, qui sera chargé de garantir que les objectifs de durabilité sont atteints tout au long du processus de développement.
    \item Utilisation de mesures d'énergie dans l'intégration continue : les auteurs suggèrent d'utiliser des mesures d'énergie dans le cadre de l'intégration continue, une pratique où les modifications de code sont régulièrement intégrées et testées automatiquement. En mesurant la consommation d'énergie pendant ce processus, les développeurs peuvent identifier les domaines où l'efficacité énergétique peut être améliorée.
    \item Mesure et surveillance des aspects de durabilité : afin de s'assurer que les objectifs de durabilité sont atteints, il est important de mesurer et de surveiller les paramètres pertinents tout au long du processus de développement logiciel. Les auteurs suggèrent d'utiliser un ensemble de mesures développées par Albertao \cite{Albertao} pour évaluer les problématiques de durabilité dans des projets logiciels réels.
\end{enumerate}
Cette approche consiste à intégrer les considérations environnementales à chaque étape du processus de développement logiciel. Cela se fait en définissant des objectifs de durabilité, en identifiant les parties prenantes, en intégrant des critères de durabilité dans les user stories, en ajoutant de nouveaux rôles et responsabilités à l'équipe Scrum, en utilisant des mesures énergétiques dans l'intégration continue, et en mesurant et en suivant les paramètres pertinents tout au long du processus.

\paragraph{Les résultats}
L'article "Green Software Engineering with Agile Methods" propose un modèle d'intégration des améliorations de l'ingénierie logicielle verte et durable dans Scrum. Les auteurs suggèrent qu'en incorporant des considérations environnementales à chaque étape du processus de développement logiciel, il est possible de créer des logiciels plus respectueux de l'environnement dès le départ.

Bien que l'article ne fournisse pas de résultats spécifiques issus de la mise en place de ce modèle, il suggère plusieurs façons de mesurer et de surveiller les aspects de durabilité tout au long du processus de développement logiciel. En utilisant un ensemble de mesures développées par Albertao \cite{Albertao}, les développeurs peuvent évaluer les problèmes de durabilité dans des projets logiciels réels et améliorer continuellement leurs produits en ce qui concerne ces questions de durabilité.

Pour finir, le résultat de cette approche devrait montrer des produits logiciels, plus respectueux de l'environnement, avec une empreinte carbone réduit et une meilleure efficacité énergétique. Cependant, comme le souligne la conclusion de l'article, il n'existe aucune preuve générale indiquant si les mesures proposées conduisent effectivement à des produits logiciels plus efficaces ou plus écologiques.
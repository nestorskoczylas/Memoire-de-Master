% Fiche nº10

\chapter{Fiche nº10} % Main appendix title
\label{app:Fiche10} % For referencing this appendix elsewhere, use \ref{app:Fiche}

%%%%%%%%%%%%%%%%%%%%%%%%%%%%%%%%%%%%%%%%%%%%%%%%%%%%%%%%%%%%%%%%%%%%%%%%%%%%%%%%%%
\section{Description de l'article}

\paragraph{Titre de l'article~: \textnormal{The Impact of Green Feedback on Users’ Software Usage}}
\paragraph{Lien de l'article~: \textnormal{https://ieeexplore.ieee.org/document/9953563}}
\paragraph{Liste des auteurs~: \textnormal{Adel Noureddine and Martín Diéguez Lodeiro and Noëlle Bru and Richard Chbeir}}
\paragraph{Nom de la conférence / revue~: \textnormal{IEEE Transactions on Sustainable Computing}}
\paragraph{Classification de la conférence / revue~: \textnormal{Rank Q1}}
\paragraph{Nombre de citations de l'article (quelle source ?)~: \textnormal{1 citation (Google Scholar)}}



%%%%%%%%%%%%%%%%%%%%%%%%%%%%%%%%%%%%%%%%%%%%%%%%%%%%%%%%%%%%%%%%%%%%%%%%%%%%%%%%%%
\section{Synthèse de l'article}

\paragraph{Problématique}
L'article "The Impact of Green Feedback on Users’ Software Usage" présente une étude sur la manière dont le fait de fournir des commentaires verts aux utilisateurs peut influencer leur comportement d'utilisation des logiciels et potentiellement réduire la consommation d'énergie. Bien que l'étude fournisse des informations précieuses sur les perceptions et les comportements des utilisateurs concernant la consommation d'énergie liée à l'utilisation des logiciels, plusieurs problèmes et limites notables doivent être résolus.

L’une des principales préoccupations de cet article est la portée limitée et la généralisabilité des résultats. L’étude se concentre principalement sur un groupe démographique spécifique de participants, à savoir les étudiants en informatique et en génie, qui peuvent ne pas être représentatifs de la population dans son ensemble. Cette focalisation étroite sur un groupe spécifique de participants limite l'applicabilité des résultats à une base d'utilisateurs plus diversifiée. De plus, l’étude reconnaît que l’échantillon comprenait des participants plus instruits et plus avertis en technologie, soulignant ainsi le biais potentiel des résultats.

Un autre problème important est le manque de clarté et de profondeur de certains aspects de l’étude. Par exemple, bien que l'article discute des perceptions des participants concernant la consommation d'énergie et de leurs réponses aux commentaires écologiques, l'exploration des raisons sous-jacentes à ces perceptions est limitée. Comprendre les facteurs qui influencent les attitudes et les comportements des utilisateurs à l'égard de la consommation d'énergie dans l'utilisation des logiciels est crucial pour développer des interventions et des stratégies efficaces. De plus, l'article mentionne que près de la moitié des participants n'ont pas compris les mesures fournies dans l'outil de feedback vert, ce qui indique une lacune potentielle dans la communication ou dans la compréhension des utilisateurs.

De plus, l'étude touche aux connaissances limitées des participants sur les stratégies d'économie d'énergie au-delà du changement de logiciel ou de la réduction du temps d'utilisation. Cette découverte souligne la nécessité d'initiatives d'éducation et de sensibilisation plus complètes pour permettre aux utilisateurs de prendre des décisions éclairées concernant l'utilisation de logiciels économes en énergie. De plus, l’étude suggère que les participants pourraient avoir été influencés par leurs cas d’utilisation spécifiques au cours de l’expérience, comme l’utilisation intensive de logiciels de communication pendant la période de confinement, ce qui aurait pu fausser leur perception de la consommation d’énergie.

De plus, l’article ne présente pas une discussion détaillée sur les implications des résultats pour les développeurs de logiciels et les décideurs politiques. Bien que l'étude souligne l'importance d'optimiser les logiciels de communication et de fournir de meilleurs mécanismes de retour d'information aux utilisateurs, elle ne parvient pas à fournir des recommandations concrètes aux acteurs de l'industrie pour promouvoir des pratiques d'utilisation durables des logiciels.

En conclusion, même si l'article met en lumière l'impact du feedback vert sur le comportement d'utilisation des logiciels des utilisateurs, il est essentiel d'aborder les limites et les lacunes identifiées pour améliorer la robustesse et la pertinence des résultats de l'étude. Les recherches futures dans ce domaine devraient s'efforcer d'atteindre un bassin de participants plus diversifié, approfondir les facteurs sous-jacents qui influencent les comportements des utilisateurs et fournir des recommandations concrètes pour promouvoir des pratiques d'utilisation de logiciels économes en énergie.

\paragraph{Pistes possibles (pointés par les auteurs)}
Les auteurs de "The Impact of Green Feedback on Users' Software Usage" suggèrent plusieurs pistes potentielles de recherche et d'action futures sur la base des résultats de leur étude :

\begin{itemize}
    \item Extension de l'étude : les auteurs recommandent d'étendre l'étude pour mieux comprendre les perceptions des utilisateurs en matière de consommation d'énergie dans l'utilisation des logiciels, en particulier dans l'optimisation des environnements de développement intégrés (IDE) et des logiciels de communication. Ils soulignent la nécessité d'explorer comment les développeurs de logiciels peuvent améliorer l'efficacité énergétique et optimiser les IDE pour réduire la consommation d'énergie.
    \item Optimisation des logiciels de communication : L’étude souligne l’importance d’optimiser les logiciels de communication, notamment dans le contexte d’une utilisation accrue pendant des périodes comme les confinements. Les auteurs suggèrent de passer à des algorithmes de codage/décodage moins gourmands en machine dans les logiciels de communication afin d’alléger la charge et la consommation d’énergie des appareils mobiles et informatiques.
    \item Intégration de la dépendance au cloud computing : les auteurs recommandent de décrire la dépendance des appareils mobiles au cloud computing dans les outils de feedback écologique. Ils suggèrent de montrer l’impact de l’ensemble de la chaîne logicielle et matérielle lorsque les utilisateurs interagissent avec leurs appareils, en particulier dans le contexte d’une dépendance croissante aux services basés sur le cloud pour diverses fonctionnalités.
    \item Rétroaction verte métaphorique : les auteurs proposent d'utiliser des métaphores, telles qu'un système de feux de circulation, dans les outils de rétroaction verte pour rendre les mesures de consommation d'énergie plus pertinentes et compréhensibles pour les utilisateurs. Ils suggèrent que l’utilisation de métaphores peut contribuer à sensibiliser les utilisateurs à la consommation électrique réelle des appareils et des logiciels.
    \item Améliorer la sensibilisation et l'éducation des utilisateurs : L'étude souligne l'importance d'améliorer les connaissances et les outils des utilisateurs pour adopter des changements de comportement durables et économes en énergie. Les auteurs suggèrent que fournir des conseils et une éducation aux utilisateurs sur les stratégies d'économie d'énergie au-delà du simple changement de logiciel ou de la réduction du temps d'utilisation est crucial pour promouvoir des pratiques d'utilisation durables des logiciels.
    \item Focus sur les utilisateurs finaux pour l'optimisation énergétique : Les auteurs plaident en faveur d'une évolution vers l'implication des utilisateurs finaux dans les efforts d'optimisation énergétique, en particulier en réduisant l'utilisation des appareils des technologies de l'information et de la communication (TIC). Ils mettent en évidence le potentiel d'économies d'énergie significatives en modifiant le comportement des utilisateurs concernant les appareils et logiciels intelligents, soulignant ainsi la nécessité d'approches de réduction d'énergie centrées sur l'utilisateur.
\end{itemize}

En mettant en évidence ces pistes de recherche et d'action futures, les auteurs visent à contribuer aux efforts en cours visant à promouvoir l'utilisation de logiciels économes en énergie et à sensibiliser les utilisateurs à l'impact environnemental de leurs comportements d'utilisation de logiciels.

\paragraph{Questions de recherche}
Les questions de recherche abordées dans l'article sont les suivantes :

\begin{itemize}
    \item Que savent les utilisateurs de la consommation énergétique des logiciels ? Quelle perception les utilisateurs ont-ils des logiciels et de l’énergie ?
    \item Le retour d’information écologique sur l’utilisation des logiciels entraînera-t-il un changement de comportement et des réductions d’énergie ?
    \item Comment les utilisateurs réagissent-ils aux outils de feedback visuel écologique ?
    \item Quels sont les changements de comportement à court terme observés chez les utilisateurs après avoir reçu un retour vert ?
\end{itemize}

Ces questions de recherche guident l'enquête sur la sensibilisation, les perceptions, les réponses et les changements de comportement des utilisateurs concernant la consommation d'énergie dans les systèmes logiciels, dans le but ultime de promouvoir des pratiques d'utilisation de logiciels durables et économes en énergie.

\paragraph{Démarche adoptée}
L'approche adoptée dans "The Impact of Green Feedback on Users' Software Usage" consiste à mener une étude de terrain pour étudier l'impact des commentaires verts sur les comportements d'utilisation des logiciels des utilisateurs. La méthodologie de recherche se concentre sur la fourniture d'un retour d'information écologique en temps réel aux utilisateurs et sur l'analyse de leurs réponses et comportements par rapport à la consommation d'énergie dans les systèmes logiciels. En utilisant une architecture distribuée pour fournir des commentaires écologiques précis et en menant des enquêtes pour recueillir les commentaires des utilisateurs, l'étude vise à évaluer l'efficacité des commentaires écologiques pour sensibiliser et susciter un changement de comportement en matière de consommation d'énergie des logiciels.

L'approche consiste à explorer les perceptions et les connaissances des utilisateurs sur la consommation d'énergie dans les logiciels, ainsi que leurs réponses aux outils de retour visuel écologique affichant des mesures telles que la consommation d'énergie, les émissions de CO2 et le prix de l'électricité. En étudiant les interactions des utilisateurs avec les mécanismes de rétroaction écologique et en analysant leurs changements de comportement à court terme, la recherche cherche à comprendre le potentiel des utilisateurs à adopter des pratiques économes en énergie dans l'utilisation des logiciels.

De plus, l'approche met l'accent sur l'optimisation des logiciels de communication et l'intégration de la dépendance au cloud computing dans les outils de feedback écologique afin de fournir aux utilisateurs une compréhension globale de l'impact énergétique de leur utilisation de logiciels. En soulignant l'importance de la sensibilisation des utilisateurs, de l'éducation et des métaphores dans les outils de feedback écologique, l'étude vise à promouvoir des comportements logiciels durables et à permettre aux utilisateurs de prendre des décisions éclairées sur les pratiques d'utilisation des logiciels économes en énergie.

Dans l'ensemble, l'approche adoptée dans l'étude combine la fourniture de commentaires écologiques en temps réel, des enquêtes auprès des utilisateurs et l'analyse des réponses des utilisateurs pour étudier l'impact des commentaires écologiques sur les comportements d'utilisation des logiciels des utilisateurs. En abordant des questions de recherche liées à la sensibilisation, aux perceptions et aux changements de comportement des utilisateurs concernant la consommation d'énergie dans les systèmes logiciels, l'étude vise à contribuer à la promotion de pratiques logicielles économes en énergie et à la durabilité environnementale.

\paragraph{Implémentation de la démarche}
L'approche présentée dans l'article est mise en œuvre à travers une expérience sur le terrain qui implique les étapes suivantes :

\begin{enumerate}
    \item Recrutement des participants : L'étude a recruté une centaine d'étudiants diplômés en informatique et en ingénierie au Liban pour participer à l'étude et à l'enquête sur le terrain. Les étudiants ont été inscrits à un cours optionnel sur les méthodologies de recherche, où ils ont été initiés aux objectifs de l'expérience.
    \item Configuration expérimentale : L'étude sur le terrain a été menée pendant des séances de classe, avec des protocoles similaires pour le groupe témoin et le groupe de traitement. Les participants ont été divisés en deux groupes, l'un recevant un retour visuel vert sur leur consommation d'énergie lors de l'utilisation du logiciel, et l'autre groupe servant de témoin.
    \item Collecte de données : L'étude a surveillé la consommation d'énergie liée à l'utilisation des logiciels des participants, en se concentrant sur des composants spécifiques tels que le processeur et les applications logicielles actives. Les données collectées comprenaient des mesures liées à la consommation d'énergie, au prix de l'électricité et aux émissions de CO2.
    \item Administration de l'enquête : Après la session expérimentale, les participants ont été invités à remplir une enquête pour fournir des commentaires sur leur expérience avec les outils de feedback écologique et leurs connaissances de la consommation d'énergie des logiciels. L'enquête visait à recueillir des informations sur les perceptions des utilisateurs et les changements de comportement suite à l'intervention de feedback vert.
    \item Analyse des résultats : Les données collectées lors de l'expérience sur le terrain et des enquêtes ont été analysées pour évaluer l'impact des commentaires verts sur les comportements d'utilisation des logiciels des utilisateurs. L'étude a évalué les réponses des utilisateurs aux outils de retour visuel écologique, leurs changements de comportement à court terme et leur conscience de la consommation d'énergie des logiciels.
    \item Recommandations et conclusions : Sur la base des résultats de l'étude de terrain, les auteurs ont présenté des recommandations et des conclusions concernant l'efficacité du feedback vert pour sensibiliser et promouvoir des changements de comportement en matière de consommation d'énergie des logiciels. L'étude a souligné l'importance de l'éducation des utilisateurs, des métaphores dans les outils de feedback et des approches d'optimisation énergétique centrées sur l'utilisateur.
\end{enumerate}

En suivant cette approche de mise en œuvre, l'étude visait à fournir un aperçu de l'impact des commentaires écologiques sur les comportements d'utilisation des logiciels des utilisateurs et à contribuer à la compréhension des pratiques économes en énergie dans les systèmes logiciels.

\paragraph{Les résultats}
Les participants ont été invités à évaluer la consommation d'énergie de leur session d'utilisation du logiciel. La majorité des participants ont évalué leur séance comme étant faible ou très faible en matière de consommation d'énergie. Cependant, les participants qui ont reçu un retour visuel vert ont évalué leur séance d'énergie plus haut que le groupe témoin, ce qui indique un impact potentiel du retour vert sur la perception.

L’étude a révélé que les commentaires écologiques contribuaient à sensibiliser les utilisateurs à la consommation d’énergie des logiciels. Les participants ont montré leur volonté d’appliquer des changements économes en énergie, mais il leur manquait les connaissances et les outils nécessaires pour adopter des changements de comportement durables et efficaces. Cela suggère un écart entre la sensibilisation et la mise en œuvre pratique des comportements économes en énergie.

La recherche a observé que les utilisateurs présentaient des changements de comportement à court terme en réponse aux commentaires verts. Cependant, les utilisateurs résistaient aux changements de comportement des logiciels, à moins que la tâche ne soit perçue comme moins importante ou qu'elle consomme peu d'énergie. Cela met en évidence l’importance de la pertinence des tâches et de l’impact énergétique perçu dans la conduite des changements de comportement.

Les utilisateurs ont interagi avec des outils de retour visuel écologique affichant des mesures telles que la consommation d'énergie, le prix de l'électricité et les émissions de CO2. L'étude a indiqué que les utilisateurs pourraient ne pas trouver de données énergétiques techniquement détaillées pertinentes pour les changements de comportement, soulignant la nécessité de mécanismes de rétroaction conviviaux et intuitifs.

L’étude comprenait des participants diplômés en informatique et en ingénierie, avec une majorité de participants masculins. L'âge moyen des participants était de 22 ans et la plupart des participants possédaient un ou deux appareils mobiles. Comprendre les données démographiques des participants a fourni un aperçu de la base d'utilisateurs pour les interventions logicielles économes en énergie.

Dans l’ensemble, les résultats de l’étude suggèrent que les commentaires écologiques peuvent sensibiliser à la consommation d’énergie des logiciels et influencer les changements de comportement à court terme parmi les utilisateurs. Cependant, les utilisateurs peuvent avoir besoin de connaissances et d'outils supplémentaires pour maintenir des comportements économes en énergie à long terme. Les résultats soulignent l’importance de l’éducation des utilisateurs, des outils de rétroaction conviviaux et de la pertinence des tâches dans la promotion de pratiques d’utilisation de logiciels économes en énergie.
% Fiche nº11

\chapter{Fiche nº11} % Main appendix title
\label{app:Fiche11} % For referencing this appendix elsewhere, use \ref{app:Fiche}

%%%%%%%%%%%%%%%%%%%%%%%%%%%%%%%%%%%%%%%%%%%%%%%%%%%%%%%%%%%%%%%%%%%%%%%%%%%%%%%%%%
\section{Description de l'article}

\paragraph{Titre de l'article~: \textnormal{Application of the Sustainability Awareness Framework in Agile Software Development}}
\paragraph{Lien de l'article~: \textnormal{https://ieeexplore.ieee.org/document/10260996}}
\paragraph{Liste des auteurs~: \textnormal{Peter Bambazek; Iris Groher; Norbert Seyff}}
\paragraph{Nom de la conférence / revue~: \textnormal{2023 IEEE 31st International Requirements Engineering Conference (RE)}}
\paragraph{Classification de la conférence / revue~: \textnormal{Rank A*}}
\paragraph{Nombre de citations de l'article (quelle source ?)~: \textnormal{2 citations (Google Scholar)}}



%%%%%%%%%%%%%%%%%%%%%%%%%%%%%%%%%%%%%%%%%%%%%%%%%%%%%%%%%%%%%%%%%%%%%%%%%%%%%%%%%%
\section{Synthèse de l'article}

\paragraph{Problématique}
L'article "Application of the Sustainability Awareness Framework in Agile Software Development" fournit des informations précieuses sur l'intégration des considérations de durabilité dans les processus de développement de logiciels agiles. Cependant, il existe certains domaines dans lesquels l'article pourrait être amélioré afin d'en accroître la clarté et la profondeur.

Il pourrait bénéficier d'une explication plus détaillée de la méthodologie utilisée pour appliquer le Sustainability Awareness Framework (SusAF) dans les études de cas. Fournir un guide étape par étape ou un aperçu clair de la manière dont les effets de durabilité ont été identifiés et liés aux éléments en retard améliorerait la reproductibilité et la compréhension du processus de recherche.

Bien que l'article mentionne brièvement les défis rencontrés lors de l'analyse de la durabilité des éléments en attente, une discussion plus approfondie sur les obstacles spécifiques rencontrés et la manière dont ils ont été surmontés fournirait des informations précieuses aux praticiens cherchant à mettre en œuvre des approches similaires dans leurs projets.

L'article aborde l'implication de différents rôles dans la création et la modification des éléments du backlog, mais une exploration plus approfondie de la manière dont les différentes parties prenantes, notamment les propriétaires de produits, les développeurs et les utilisateurs finaux, ont contribué au processus d'analyse de la durabilité offrirait une vue plus complète de la situation. efforts de collaboration requis dans le développement de logiciels durables.

Bien que l'article présente les conclusions générales et les résultats des études de cas, l'incorporation d'exemples plus concrets ou de cas spécifiques où les effets de durabilité ont été identifiés, mappés aux éléments en retard et traités dans le processus de développement rendrait le contenu plus attrayant et plus pratique pour les lecteurs.

L’article se concentre principalement sur les effets immédiats en matière de durabilité identifiés au cours de la phase de développement. L'inclusion d'une discussion sur l'impact à long terme sur la durabilité de la prise en compte de ces effets dans les systèmes logiciels après le déploiement fournirait une vision plus globale des avantages de l'intégration de la durabilité dans les pratiques agiles.

En conclusion, même si l'article offre des informations précieuses sur l'application du Sustainability Awareness Framework dans le développement de logiciels agiles, il améliore la description de la méthodologie, discute plus en détail des défis de mise en œuvre, explore l'implication des parties prenantes, fournit des exemples concrets et considère l'impact sur la durabilité à long terme. enrichirait davantage l’article et le rendrait plus informatif pour les chercheurs et les praticiens du domaine.

\paragraph{Pistes possibles (pointés par les auteurs)}
Les auteurs de "Application of the Sustainability Awareness Framework in Agile Software Development" suggèrent plusieurs pistes possibles pour de futures recherches et mises en œuvre sur la base de leurs résultats :

\begin{enumerate}
    \item Intégration de SusAF dans les processus agiles : les auteurs soulignent la nécessité d'approfondir les recherches sur la manière dont le Sustainability Awareness Framework (SusAF) peut être intégré efficacement dans les processus de développement logiciel agile (ASD). Ils soulignent l’importance de ne pas considérer la durabilité comme un événement ponctuel, mais plutôt de l’intégrer comme une considération continue tout au long du cycle de développement.
    \item Améliorer la sensibilisation à la durabilité dans la communauté agile : les auteurs mentionnent les commentaires positifs reçus des praticiens concernant l'utilisation de SusAF et les résultats des ateliers sur la durabilité. Ils suggèrent d'étendre davantage la collaboration en se concentrant sur les questions de durabilité et de sensibiliser les participants à l'atelier à la manière dont les systèmes logiciels sont développés et utilisés.
    \item Améliorer la qualité des éléments du backlog : les auteurs notent que seule une minorité des éléments du backlog sont rédigés sous la forme de user stories, et certains manquent de descriptions détaillées. Ils soulignent l’importance d’améliorer la qualité des descriptions des éléments de l’arriéré pour faciliter les discussions sur les effets potentiels en matière de durabilité et garantir la clarté sur la manière dont les considérations de durabilité sont liées à des éléments spécifiques de l’arriéré.
    \item Analyse détaillée des éléments de l'arriéré : Les auteurs suggèrent que les recherches futures pourraient se concentrer sur l'analyse détaillée des éléments de l'arriéré pour évaluer si les questions d'orientation fournies par SusAF pourraient être étendues en conséquence. Cela impliquerait un examen plus approfondi de la manière dont les effets de durabilité sont liés aux éléments en retard et de la manière dont ils peuvent être traités dans le processus de développement.
    \item Relever les défis de l'intégration de la durabilité : les auteurs discutent des défis rencontrés lors de la cartographie des éléments de l'arriéré en termes d'effets de durabilité, tels que la difficulté d'analyser les éléments de l'arriéré déjà mis en œuvre en matière de durabilité. Ils soulignent la nécessité de relever ces défis pour améliorer l’efficacité de l’intégration de la durabilité dans les pratiques de développement de logiciels agiles.
\end{enumerate}

En soulignant ces pistes possibles de recherche et de mise en œuvre futures, les auteurs visent à contribuer aux efforts en cours visant à intégrer de manière efficace et durable les considérations de durabilité dans les processus de développement de logiciels agiles.

\paragraph{Questions de recherche}
Les questions de recherche exposées dans cet article sont les suivantes :

\begin{itemize}
    \item Comment une approche RE établie comme SusAF peut-elle être utilisée dans l'ASD pour identifier les effets possibles sur la durabilité d'un système logiciel ?
    \item Dans quelle mesure est-il possible de relier les éléments en retard aux effets de durabilité identifiés ?
\end{itemize}

Ces questions de recherche guident l'étude dans l'examen de l'application de SusAF dans des contextes agiles, dans la compréhension des liens entre les effets de durabilité et les éléments de retard, et dans l'exploration de la manière dont les considérations de durabilité peuvent être intégrées efficacement dans le processus de développement logiciel agile.

\paragraph{Démarche adoptée}
L'approche adoptée implique l'utilisation du Sustainability Awareness Framework (SusAF) pour identifier les effets potentiels sur la durabilité des systèmes logiciels dans les processus de développement logiciel agile (ASD). L'étude se concentre sur l'intégration des considérations de durabilité dans des cadres agiles comme Scrum et souligne l'importance de l'identification précoce des effets de durabilité pour soutenir le développement de systèmes logiciels plus durables. L'approche consiste à organiser des ateliers avec les propriétaires de produits et les parties prenantes pour identifier les effets de durabilité sur la base des visions du produit et relier ces effets aux éléments en retard dans le processus de développement. En appliquant SusAF à l'ASD, l'étude vise à améliorer la sensibilisation à la durabilité, à faciliter la prise de décision concernant les questions de durabilité et, à terme, à contribuer au développement de systèmes logiciels durables dans des environnements agiles.

\paragraph{Implémentation de la démarche}
La démarche mise en œuvre comporte les étapes suivantes :

\begin{enumerate}
    \item Identifier les effets de durabilité : l'étude commence par organiser des ateliers avec les propriétaires de produits et les parties prenantes pour identifier les effets potentiels de durabilité des systèmes logiciels en fonction des visions des produits. Le Sustainability Awareness Framework (SusAF) est utilisé comme ligne directrice pour guider les discussions sur les effets de la durabilité dans diverses dimensions.
    \item Cartographie des effets de durabilité sur les éléments du backlog : les effets de durabilité identifiés sont ensuite mappés sur les éléments du backlog de produits des systèmes logiciels. Cette cartographie vise à établir des liens entre les considérations de durabilité et les tâches de développement spécifiques décrites dans le backlog.
    \item Analyse de la relation : L'étude analyse dans quelle mesure les effets de durabilité identifiés sont liés aux éléments du carnet de commandes. En examinant le chevauchement entre les effets de durabilité et les éléments de retard, la recherche vise à comprendre comment les considérations de durabilité peuvent être intégrées dans le processus de développement agile.
    \item Évaluation et réflexion : les résultats de l'exercice de cartographie sont évalués pour déterminer l'impact de l'intégration des considérations de durabilité dans le processus de développement logiciel agile. L'étude réfléchit aux résultats pour évaluer l'efficacité de l'approche pour identifier et traiter les effets de durabilité dans les projets agiles.
\end{enumerate}

En suivant ces étapes, l'étude met en œuvre le Sustainability Awareness Framework dans les processus de développement de logiciels agiles pour identifier, analyser et intégrer les considérations de durabilité dans le développement de systèmes logiciels.

\paragraph{Les résultats}
Grâce à des ateliers utilisant le Sustainability Awareness Framework (SusAF), plus de 20 effets potentiels en matière de durabilité ont été identifiés pour chacun des systèmes logiciels en fonction de leurs visions de produits. Ces effets couvrent les dimensions environnementales, économiques, techniques, sociales et individuelles, mettant en évidence les diverses considérations de durabilité dans le développement de logiciels.

L'étude a révélé que plus de la moitié des éléments en retard dans les deux études de cas pourraient être liés à au moins un des effets de durabilité identifiés. Cela suggère un chevauchement important entre les considérations de durabilité et les tâches de développement spécifiques décrites dans le backlog du produit, indiquant le potentiel d'intégration de la durabilité dans les pratiques de développement logiciel agiles.

Malgré la cartographie réussie de nombreux effets de durabilité sur les éléments de l’arriéré, certains effets n’ont pu être liés à aucun élément spécifique de l’arriéré. Ce défi souligne la nécessité d’affiner davantage la manière dont les considérations de durabilité sont intégrées dans le processus de développement agile et dont elles se reflètent dans les éléments du backlog.

Les participants, qui étaient des praticiens industriels sans aucune connaissance préalable de la durabilité dans le développement de logiciels, ont pu comprendre et s'engager dans les dimensions de la durabilité présentées par SusAF. Cet accueil positif indique le potentiel des praticiens à tirer parti de SusAF pour sensibiliser à la durabilité dans les projets de logiciels industriels.

En mettant en évidence ces résultats, l'étude démontre la faisabilité et les avantages potentiels de l'intégration de considérations de durabilité dans les processus de développement de logiciels agiles à l'aide du Sustainability Awareness Framework.
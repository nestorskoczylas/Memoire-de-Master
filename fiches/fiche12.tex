% Fiche nº12

\chapter{Fiche nº12} % Main appendix title
\label{app:Fiche12} % For referencing this appendix elsewhere, use \ref{app:Fiche}

%%%%%%%%%%%%%%%%%%%%%%%%%%%%%%%%%%%%%%%%%%%%%%%%%%%%%%%%%%%%%%%%%%%%%%%%%%%%%%%%%%
\section{Description de l'article}

\paragraph{Titre de l'article~: \textnormal{From Sustainability in Requirements Engineering to a Sustainability-Aware Scrum Framework}}
\paragraph{Lien de l'article~: \textnormal{https://ieeexplore.ieee.org/document/9604667}}
\paragraph{Liste des auteurs~: \textnormal{Peter Garscha}}
\paragraph{Affiliation des auteurs~: \textnormal{Institute of Business Informatics - Software Engineering, Johannes Kepler University Linz, Linz, Austria}}
\paragraph{Nom de la conférence / revue~: \textnormal{2021 IEEE 29th International Requirements Engineering Conference (RE)}}
\paragraph{Classification de la conférence / revue~: \textnormal{Rank A*}}
\paragraph{Nombre de citations de l'article (quelle source ?)~: \textnormal{6 citations (Google Scholar)}}



%%%%%%%%%%%%%%%%%%%%%%%%%%%%%%%%%%%%%%%%%%%%%%%%%%%%%%%%%%%%%%%%%%%%%%%%%%%%%%%%%%
\section{Synthèse de l'article}

\paragraph{Problématique}
L'article "From Sustainability in Requirements Engineering to a Sustainability-Aware Scrum Framework" de Peter Garscha aborde le sujet important de l'intégration des considérations de durabilité dans les processus de développement logiciel, en se concentrant spécifiquement sur l'ingénierie des exigences et le cadre Scrum. Bien que le document présente une direction de recherche prometteuse, plusieurs domaines pourraient être améliorés ou élargis pour accroître la qualité et l’impact global de l’étude.

L’un des principaux problèmes de cet article est le manque de profondeur dans la discussion des méthodes et cadres spécifiques qui ont été proposés pour aborder la durabilité dans l’ingénierie des exigences. Bien que le document mentionne que diverses approches ont été suggérées, il ne fournit pas un aperçu complet ni une analyse critique de ces méthodes existantes. Un examen plus détaillé de ces méthodes, y compris leurs forces, leurs limites et leur applicabilité dans différents contextes, aurait ajouté de la valeur à la discussion et contribué à éclairer le développement d'un cadre Scrum soucieux de la durabilité.

En outre, l’article pourrait bénéficier d’une exploration plus approfondie des défis et obstacles potentiels associés à l’intégration de la durabilité dans les pratiques de développement agiles comme Scrum. Bien que le document aborde brièvement les questions liées à l'intégration de la durabilité dans les réunions de planification de sprint et aux critères d'acceptation, une analyse plus approfondie des implications pratiques et des compromis impliqués dans ce processus aurait été précieuse. Comprendre les conflits potentiels entre les objectifs de développement durable et les pratiques agiles traditionnelles est crucial pour développer des stratégies efficaces de développement de logiciels soucieux du développement durable.

De plus, l’article manque de preuves empiriques ou d’études de cas pour étayer ses arguments et hypothèses. Bien que l'auteur mentionne des projets d'entretiens avec des praticiens et des études de cas à l'avenir, le document actuel aurait été renforcé en incluant des exemples ou des données du monde réel pour illustrer la faisabilité et l'impact d'un cadre Scrum soucieux de la durabilité. Fournir des exemples concrets de la manière dont les considérations de durabilité ont été intégrées avec succès dans des projets de développement agiles aurait rendu la recherche plus convaincante et plus exploitable pour les praticiens du domaine.

En conclusion, bien que l'article de Peter Garscha présente une direction de recherche intéressante sur la durabilité dans l'ingénierie des exigences et Scrum, il existe des possibilités d'amélioration en termes de profondeur d'analyse, d'exploration des défis et d'inclusion de preuves empiriques. En abordant ces domaines, l'étude pourrait améliorer sa pertinence et sa valeur pratique pour les chercheurs et les praticiens intéressés par les pratiques de développement de logiciels soucieuses du développement durable.

\paragraph{Pistes possibles (pointés par les auteurs)}
Les auteurs soulignent plusieurs pistes possibles pour de futures recherches et développements dans le domaine du développement de logiciels soucieux du développement durable. Ces pistes comprennent :

\begin{enumerate}
    \item Développement d'un cadre Scrum soucieux de la durabilité : les auteurs proposent le développement d'un cadre qui intègre les aspects de durabilité dans la méthodologie Scrum pour améliorer les tâches d'ingénierie des exigences telles que l'élicitation, l'analyse, la spécification, la validation et la gestion des exigences logicielles. Ce cadre vise à accroître la durabilité du développement de logiciels en tirant parti des avantages des pratiques de développement agiles.
    \item Enquête sur les pratiques de développement agiles pour les objectifs de durabilité : les auteurs suggèrent d'explorer comment les pratiques de développement agiles, en particulier celles du cadre Scrum, peuvent être utilisées pour atteindre les objectifs de durabilité dans le développement de logiciels. En identifiant et en adaptant les méthodes existantes traitant de la durabilité, la recherche vise à vérifier quels avantages des pratiques agiles peuvent être exploités pour améliorer les impacts sur la durabilité.
    \item Examen des impacts sur la durabilité dans les événements et artefacts Scrum : les auteurs soulignent l'importance de traiter les impacts sur la durabilité dans différents événements du flux de travail Scrum, tels que l'affinement du backlog, l'examen des produits et les rétrospectives. En incorporant des considérations de durabilité dans les artefacts Scrum tels que les témoignages d'utilisateurs, les définitions de critères de préparation et d'acceptation, la recherche vise à améliorer l'identification, l'analyse, la documentation, la validation et la gestion des impacts de durabilité des systèmes logiciels.
    \item Méthodologie de recherche en science du design : les auteurs proposent d'utiliser une méthodologie de recherche en science du design pour guider le développement du cadre Scrum soucieux de la durabilité. Cette méthodologie implique de définir des questions de recherche, des hypothèses et des exigences pour le cadre envisagé au moyen de questionnaires, d'entretiens, d'ateliers et de collaboration avec des praticiens. En suivant une approche structurée, la recherche vise à créer une solution pratique qui répond aux défis du monde réel en matière de développement de logiciels soucieux du développement durable.
\end{enumerate}

Dans l’ensemble, les auteurs suggèrent que les recherches futures devraient se concentrer sur l’intégration des considérations de durabilité dans les pratiques de développement agile, en particulier dans le cadre Scrum, afin d’améliorer les impacts des systèmes logiciels sur la durabilité. En explorant ces pistes, la recherche vise à contribuer à l’avancement des pratiques de développement de logiciels durables et à promouvoir la sensibilisation à la durabilité parmi les équipes de développement de logiciels et les parties prenantes.

\paragraph{Questions de recherche}
Les questions de recherche décrites dans l’article sont les suivantes :

\begin{itemize}
    \item Comment la durabilité peut-elle être soutenue en utilisant des pratiques de développement agiles issues de frameworks comme Scrum ?
    \item Est-il productif de prendre en compte la durabilité lors des réunions de planification de sprint ou des revues de produits ?
    \item Est-il utile d'ajouter des aspects de durabilité aux critères d'acceptation des user stories ou à une définition du fait ?
\end{itemize}

Ces questions de recherche servent de base à l'étude doctorale en cours décrite dans l'article, visant à étudier l'intégration des considérations de durabilité dans les pratiques de développement de logiciels agiles, en particulier dans le cadre Scrum. En abordant ces questions, la recherche cherche à développer un cadre Scrum soucieux de la durabilité, inspiré des méthodes d'ingénierie des exigences existantes pour améliorer les impacts sur la durabilité des systèmes logiciels et promouvoir des pratiques de développement logiciel durables.

\paragraph{Démarche adoptée}
L'approche adoptée dans l'article implique l'utilisation d'une méthodologie de recherche en science de la conception pour développer un cadre Scrum soucieux de la durabilité. Cette méthodologie se compose de cinq étapes principales et vise à créer une solution pratique pour relever les défis du monde réel en matière de développement de logiciels soucieux du développement durable dans le contexte de pratiques agiles comme Scrum. L'approche vise à concevoir et à mettre en œuvre un cadre qui intègre des considérations de durabilité dans la méthodologie Scrum pour améliorer les impacts sur la durabilité des systèmes logiciels et promouvoir la sensibilisation à la durabilité parmi les équipes de développement de logiciels et les parties prenantes.

\paragraph{Implémentation de la démarche}
L’approche décrite dans l’article est mise en œuvre à travers une série d’étapes guidées par une méthodologie de recherche en science du design. La mise en œuvre implique les activités clés suivantes :

\begin{enumerate}
    \item Définir des questions et des hypothèses de recherche : les questions et hypothèses de recherche liées à la durabilité dans les pratiques d'ingénierie des exigences et de développement de logiciels agiles sont établies pour guider le développement du cadre Scrum soucieux de la durabilité.
    \item Obtenir les exigences au moyen de questionnaires, d'entretiens et d'ateliers : les exigences du cadre envisagé sont recueillies grâce à des interactions avec des praticiens, notamment des développeurs de logiciels, des Scrum masters, des propriétaires de produits et des analystes commerciaux. Ces parties prenantes fournissent des informations et des commentaires sur les aspects de durabilité qui devraient être intégrés dans le cadre Scrum.
    \item Conception et développement du cadre Scrum soucieux du développement durable : Le cadre est conçu sur la base des exigences identifiées et des idées des parties prenantes. Des méthodes créatives telles que le brainstorming, la conception en binôme et les évaluations par les pairs sont utilisées pour générer des idées et affiner le cadre.
    \item Démonstration du cadre avec des études de cas : Le cadre Scrum développé, soucieux de la durabilité, est testé et démontré dans des contextes pratiques, comme dans une société de conseil informatique en Autriche. Des études de cas sont menées pour évaluer l'efficacité du cadre dans différents contextes organisationnels, notamment les organisations gouvernementales, les petites et moyennes entreprises et les grandes entreprises.
    \item Évaluation et raffinement itératifs : le cadre est évalué de manière itérative au moyen d'entretiens, de questionnaires et de techniques d'observation avec les participants des équipes Scrum. Les commentaires sont recueillis pour identifier les domaines à améliorer, et le cadre est affiné en fonction des résultats de l'évaluation.
\end{enumerate}

En suivant cette approche de mise en œuvre, la recherche vise à créer un cadre Scrum pratique et efficace, soucieux de la durabilité, capable d'améliorer les impacts sur la durabilité des systèmes logiciels et de soutenir une prise de décision éclairée dans les pratiques de développement logiciel agiles.

\paragraph{Les résultats}
L'article ne fournit pas de résultats ou de conclusions spécifiques car il décrit une étude doctorale en cours visant à développer un cadre Scrum soucieux de la durabilité. La recherche présentée dans l'article se concentre sur la proposition de questions de recherche, d'hypothèses et d'une méthodologie pour concevoir et mettre en œuvre le cadre plutôt que de présenter des résultats ou des résultats empiriques. L'étude est toujours en cours et les auteurs prévoient de mener des études de cas et des évaluations pour évaluer l'efficacité du cadre Scrum soucieux de la durabilité dans différents contextes organisationnels.
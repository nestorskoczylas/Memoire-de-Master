% Fiche nº2

\chapter{Fiche nº2} % Main appendix title
\label{app:Fiche2} % For referencing this appendix elsewhere, use \ref{app:Fiche2}

%%%%%%%%%%%%%%%%%%%%%%%%%%%%%%%%%%%%%%%%%%%%%%%%%%%%%%%%%%%%%%%%%%%%%%%%%%%%%%%%%%
\section{Description de l'article}

\paragraph{Titre de l'article~: \textnormal{The GREENSOFT Model: A reference model for green and sustainable software and its engineering}}
\paragraph{Lien de l'article~: \textnormal{https://www.sciencedirect.com/science/article/abs/pii/S2210537911000473}}
\paragraph{Liste des auteurs~: \textnormal{Stefan Naumann, Markus Hirsch-Dick, Eva Kern, Timo Johann}}
\paragraph{Affiliation des auteurs~: \textnormal{Trier University of Applied Sciences, Umwelt-Campus Birkenfeld (Environmental Campus Birkenfeld), ISS - Institute for Software Systems, P.O. Box 1380, 55761 Birkenfeld, Germany}}
\paragraph{Nom de la conférence / revue~: \textnormal{Sustainable Computing: Informatics and Systems}}
\paragraph{Classification de la conférence / revue~: \textnormal{Q1}}
\paragraph{Nombre de citations de l'article (quelle source ?)~: \textnormal{244 citations(Semantic Scholar)}}

%%%%%%%%%%%%%%%%%%%%%%%%%%%%%%%%%%%%%%%%%%%%%%%%%%%%%%%%%%%%%%%%%%%%%%%%%%%%%%%%%%
\section{Synthèse de l'article}

\paragraph{Problématique}
L’article “The GREENSOFT Model: A reference model for green and sustainable software and its engineering" aborde la problématique de l'impact environnemental et social du développement et de l'utilisation de logiciels. Au fil des ans, la demande croissante de produits et de services logiciels a entraîné une consommation d'énergie exponentielle, des émissions massives de carbone et une augmentation importante des déchets électroniques. L'article propose une façon de repenser l'approche du développement et de l'utilisation des logiciels pour y minimiser l'impact négatif sur l'environnement et la société.

C'est dans cette optique qu'il introduit plusieurs aspects clé. Tout d'abord, il souligne l'impact environnemental lié au développement de logiciels, comme la consommation d'énergie élevée tout au long de leur cycle de vie. Des pratiques durables nécessitent d'être adoptées, en commençant dès la phase d'ingénierie des exigences, en intégrant des critères d'efficacité énergétique.

Ensuite, l'article évoque les aspects de durabilité sociale du développement de logiciels. Le développement durable, vise à satisfaire les besoins présents sans compromettre la capacité des générations futures à satisfaire les leurs. Il exige donc de considérer l'accessibilité des logiciels à tous les utilisateurs, indépendamment de leurs capacités ou de leurs handicaps.

Un autre point essentiel soulevé par l'article est la nécessité de la collaboration des parties prenantes du développement et de l'utilisation de logiciels. Les développeurs, les administrateurs, les utilisateurs et d'autres acteurs doivent donc travailler ensemble pour créer, maintenir et utiliser des logiciels de manière durable. Le modèle GREENSOFT fournit ainsi des modèles de procédure adaptés à chaque partie prenante.

De plus, l'article met en lumière le besoin de critères et de métriques de durabilité afin de mesurer l'impact environnemental des produits logiciels tout au long de leur cycle de vie. Les indicateurs proposés par le modèle GREENSOFT incluent la consommation d'énergie, les émissions de carbone, la consommation d'eau, le taux de recyclage, de réutilisation, de réparation, de mise à niveau, et d'élimination des matériaux, entre autres. Ces mesures permettent donc aux parties prenantes d'identifier les domaines dans lesquels elles peuvent améliorer la durabilité de leurs produits et services logiciels.

Par ailleurs, l'article souligne également l'importance des politiques d'approvisionnement durables pour les produits et services logiciels. Ces politiques visent à garantir que les logiciels sont achetés auprès de fournisseurs répondant à certains critères de durabilité. Le modèle GREENSOFT propose également des lignes directrices pour le développement de telles politiques, en abordant des critères environnementaux, sociaux et économiques. Ainsi, il s'agit d'instaurer une approche en faveur de l'acquisition de solutions logicielles durables.

Un autre aspect est abordé dans l'article et concerne les modèles de licences durables pour les produits et services logiciels. Ces modèles visent à garantir que les logiciels sont distribués de manière à promouvoir la durabilité.

Enfin, l'article met en évidence l'importance des outils qui soutiennent les parties prenantes dans la mise en œuvre de pratiques d'ingénierie logicielle durables. Le modèle GREENSOFT propose une gamme d'outils tels que des outils de codage économes en énergie, des calculateurs d'empreinte carbone et des outils d'évaluation de la durabilité.

\paragraph{Pistes possibles (pointés par les auteurs)}
Les auteurs de l'article proposent plusieurs pistes pour promouvoir des pratiques d'ingénierie logicielle verte et durable. Ils recommandent d'intégrer des critères d'efficacité énergétique dès la phase d'ingénierie des exigences. Ils soulignent également l'importance de considérer l'accessibilité des logiciels à tous les utilisateurs, indépendamment de leurs capacités. Ils encouragent la collaboration entre les parties prenantes du développement et de l'utilisation de logiciels, en mettant en place des procédures pour les processus durable de développement, d'approvisionnement, d'exploitation et de maintenance, ainsi que la fin de vie. De même, ils recommandent l'utilisation de critères et de métriques de durabilités pour mesurer l'impact environnemental des logiciels tout au long de leur cycle de vie.

\paragraph{Questions de recherche}
\begin{itemize}
    \item Quel est l'impact du génie logiciel sur le développement durable, et comment rendre les pratiques de génie logiciel plus durable ?
    \item Quels sont les défis et les possibilités de mettre en place des pratiques durables de génie logiciel, et comment peut-on relever ces défis ?
    \item Quels sont les principaux paramètres et critères de durabilité qui devraient être pris en compte à toutes les étapes du cycle de développement du logiciel, et comment ces paramètres peuvent-ils être mesurés ?
    \item Comment les intervenants qui participent au développement et à l'utilisation de logiciels peuvent-ils collaborer pour s'assurer qu'ils créent, maintiennent et utilisent des logiciels de façon plus durable ?
    \item Quels outils et cadres peuvent être élaborés pour aider les intervenants à mettre en œuvre des pratiques durable de génie logiciel ?
\end{itemize}

\paragraph{Démarche adoptée}
Les auteurs proposent le modèle GREENSOFT, qui est un modèle de référence conceptuel pour les logiciels verts et durables. Ce modèle fournit des lignes directrices et des outils pour créer, maintenir et utiliser des logiciels de manière plus durable. Les pistes possibles identifiées par les auteurs incluent :
\begin{enumerate}
    \item Élaborer un modèle de cycle de vie holistique pour les produits logiciels qui tient compte des paramètres et des critères de durabilité tout au long du cycle de vie.
    \item Adopter des pratiques de codage éco-énergétiques pendant le processus de développement de logiciels afin de réduire la consommation d'énergie.
    \item Élaborer des politiques d'approvisionnement durable qui garantissent que les produits et services logiciels sont achetés auprès de fournisseurs qui répondent à certains critères du durabilité.
    \item Élaborer des modèles de licences durables qui favorisent la durabilité.
    \item Fournir des outils qui aident les intervenants à mettre en place des pratiques de génie logiciel durables.
    \item Collaborer avec les différentes parties prenantes impliquées dans le développement et l'utilisation de logiciels pour s'assurer qu'ils créent, maintiennent et utilisent des logiciels de manière plus durable.
    \item Tenir compte des aspects liés à la durabilité sociale.
\end{enumerate}

\paragraph{Implémentation de la démarche}
L'approche proposée dans cet article implique le développement d'un modèle appelé GREENSOFT, qui fournit des lignes directrices et des outils. Ce modèle comprend quatre parties principales :
\begin{enumerate}
    \item Cycle de vie d'un produit logiciel - Cette première partie du modèle tient compte des paramètres et des critères de durabilité tout au long du cycle de vie des produits logiciels, de la conception et du développement à l'utilisation et à l'élimination.
    \item Critères et paramètres pour les effets directs et indirects des logiciels sur le développement durable - Cette deuxième partie du modèle apporte des lignes directrices pour mesurer les indicateurs de durabilité tels que la consommation d'énergie, l'empreinte carbone, la consommation de ressources, etc.
    \item Modèles de procédures pour le développement, l'achat, l'exploitation et l'utilisation de logiciels durable - Cette troisième partie du modèle procure des lignes directrices pour l'adoption de pratiques durables à toutes les étapes du cycle de développement de logiciels.
    \item Cadre de recommandations d'actions et d'outils - Cette quatrième et dernière partie du modèle offre un ensemble d'outils qui aident les intervenants à mettre en place des pratiques durable de génie logiciel. Ces outils comprennent des outils de codage éco-énergétiques, des calculateurs d'empreinte carbone, des outils d'évaluation de la durabilité, etc.
\end{enumerate}

\paragraph{Les résultats}
L'article propose une approche globale des pratiques de génie logiciel durable, qui comprend un modèle de cycle de vie holistique pour les produits logiciels, des critères de durabilités et des mesures, des modèles de procédures pour les différentes parties prenantes, des recommandations d'action, ainsi que des outils qui aident les parties prenantes à développer, acheter, fournir et utiliser les logiciels de manière plus écologique et plus durable.

Les auteurs, eux, suggèrent qu’en adoptant cette approche et en mettant en œuvre ses lignes directrices et recommandations, les parties prenantes peuvent réduire leur impact environnemental tout en améliorant leurs aspects de durabilité sociale. La mise en œuvre implique une collaboration avec les différentes parties prenantes impliquées dans le développement et l’utilisation de logiciels pour s’assurer qu’ils créent, maintiennent et utilisent des logiciels d’une manière plus durable.

Le résultat de cet article est un cadre proposé, pour atteindre des pratiques durables de génie logiciel qui peuvent aider à réduire l’impact environnemental du développement et de l’utilisation de logiciels tout en améliorant les aspects de durabilité sociale.

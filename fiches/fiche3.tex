% Fiche nº3

\chapter{Fiche nº3} % Main appendix title
\label{app:Fiche3} % For referencing this appendix elsewhere, use \ref{app:Fiche3}

%%%%%%%%%%%%%%%%%%%%%%%%%%%%%%%%%%%%%%%%%%%%%%%%%%%%%%%%%%%%%%%%%%%%%%%%%%%%%%%%%%
\section{Description de l'article}

\paragraph{Titre de l'article~: \textnormal{An Empirical Study of Practitioners' Perspectives on Green Software Engineering}}
\paragraph{Lien de l'article~: \textnormal{https://dl.acm.org/doi/abs/10.1145/2884781.2884810}}
\paragraph{Liste des auteurs~: \textnormal{Irene Manotas, Christian Bird, Rui Zhang, David Shepherd, Ciera Jaspan, Caitlin Sadowski, Lori Pollock, James Clause}}
\paragraph{Affiliation des auteurs~: \textnormal{University of Delaware, Newark, DE, USA., Microsoft Research, Redmond, WA, USA., IBM Research - Almaden, San Jose, CA, USA., ABB Corporate Research, Raleigh, NC, USA., Google, Inc., Mountain View, CA, USA.}}
\paragraph{Nom de la conférence / revue~: \textnormal{2016 IEEE/ACM 38th IEEE International Conference on Software Engineering}}
\paragraph{Classification de la conférence / revue~: \textnormal{A*}}
\paragraph{Nombre de citations de l'article (quelle source ?)~: \textnormal{73 citations(ACM Digital Library)}}



%%%%%%%%%%%%%%%%%%%%%%%%%%%%%%%%%%%%%%%%%%%%%%%%%%%%%%%%%%%%%%%%%%%%%%%%%%%%%%%%%%
\section{Synthèse de l'article}

\paragraph{Problématique}
La problématique abordée dans l'article "An Empirical Study of Practitioners' Perspectives on Green Software Engineering" porte sur l'ingénierie logicielle écologique et la nécessité de prendre en compte l'impact environnemental du développement de logiciels. Alors que l'utilisation croissante de la technologie est devenue omniprésente dans notre vie quotidienne, il est essentiel de se pencher sur les pratiques actuelles des ingénieurs logiciels en termes de consommation d'énergie et d'émissions de carbone associées.

L'une des principales questions abordées dans cet article est le manque de connaissances sur les pratiques et les perspectives actuelles des ingénieurs logiciels dans ce domaine. Alors que la recherche sur l'ingénierie logicielle écologique se développe, on sait peu de choses sur la façon dont les praticiens pensent à l'énergie lorsqu'ils développent des logiciels. Cette étude vise à combler cette lacune en donnant un aperçu de la manière dont les praticiens abordent la consommation d'énergie dans leur travail.

Un autre point mis en évidence dans cet article est la nécessité de mener davantage de recherches sur l'ingénierie logicielle écologique. Bien que cette étude fournisse des informations précieuses sur les pratiques et les perspectives actuelles, il reste encore beaucoup à apprendre sur la manière de développer des logiciels plus durables. Les auteurs suggèrent que les recherches futures se concentrent sur le développement des meilleures pratiques en matière d'ingénierie logicielle écologique.

L'article aborde également certaines idées reçues sur la consommation d'énergie dans le développement de logiciels. Par exemple, de nombreuses personnes supposent que les services basés sur le cloud sont plus respectueux de l'environnement que les centres de données traditionnels parce qu'ils sont plus efficaces. Toutefois, cela n'est pas toujours vrai, car les services en nuage nécessitent souvent plus de matériel que les centres de données traditionnels.

Les auteurs mettent également en évidence certaines bonnes pratiques en matière d'ingénierie logicielle écologique. Ils suggèrent que les développeurs prennent en compte l'efficacité énergétique lors de la conception des algorithmes et des structures de données. Ils recommandent également d'utiliser des outils tels que le profilage et la surveillance pour identifier le code à forte consommation d'énergie et l'optimiser en termes d'efficacité énergétique.

\paragraph{Pistes possibles (pointés par les auteurs)}
Les auteurs de l'article soulignent plusieurs pistes pour l'ingénierie logicielle écologique. Ils mettent en avant la nécessité d'intégrer l'efficacité énergétique dès la conception des algorithmes et des structures de données. En concevant des algorithmes plus efficaces et en utilisant des structures de données optimisées, il est possible de réduire la consommation d'énergie du logiciel. Les auteurs recommandent également l'utilisation d'outils de profilage et de surveillance pour identifier les parties du code qui consomment le plus d'énergie, afin de les optimiser en termes d'efficacité énergétique. Par ailleurs, ils soulignent l'importance de sensibiliser davantage les praticiens aux enjeux environnementaux liés au développement de logiciels et de leur fournir les connaissances et les compétences nécessaires pour adopter des approches plus durables. En encourageant la recherche continue dans ce domaine, les auteurs suggèrent le développement de meilleures pratiques en matière d'ingénierie logicielle écologique afin de promouvoir des logiciels plus respectueux de l'environnement.

\paragraph{Questions de recherche}
\begin{itemize}
    \item Quelles sont les pratiques actuelles des ingénieurs logiciels en termes de prise en compte de l'efficacité énergétique lors du développement de logiciels ?
    \item Quel est l'impact environnemental du développement de logiciels, notamment en ce qui concerne la consommation d'énergie et les émissions de carbone ?
    \item Quelles sont les meilleures pratiques en matière d'ingénierie logicielle écologique pour réduire la consommation d'énergie et les émissions de carbone ?
    \item Quelle est l'efficacité énergétique des services basés sur le cloud par rapport aux centres de données traditionnels ?
    \item Comment sensibiliser et former les praticiens de l'ingénierie logicielle aux enjeux environnementaux et leur fournir les compétences nécessaires pour adopter des pratiques plus durables ?
\end{itemize}

\paragraph{Démarche adoptée}
Afin de pallier ces problématiques, les auteurs adoptent une approche mixte pour étudier les pratiques et les perspectives actuelles des ingénieurs logiciels dans le domaine de l'ingénierie logicielle écologique. L'étude utilise des méthodes qualitatives et quantitatives pour recueillir des données auprès de praticiens d'ABB, de Google, d'IBM et de Microsoft.
\begin{itemize}
    \item Le volet qualitatif de l'étude comprend des entretiens approfondis avec 18 employés de Microsoft. Ces entretiens permettent de comprendre comment les praticiens pensent à l'énergie lorsqu'ils développent des logiciels. Les auteurs utilisent ces informations pour élaborer une enquête ciblée qui est distribuée à 464 praticiens dans les quatre entreprises.
    \item Le volet quantitatif de l'étude consiste à analyser les réponses à l'enquête afin d'identifier les thèmes communs et les tendances dans la manière dont les praticiens abordent la consommation d'énergie dans leur travail. Les auteurs utilisent des techniques d'analyse statistique pour identifier les corrélations entre différentes variables, telles que la taille de l'entreprise et la consommation d'énergie.
\end{itemize}
En utilisant à la fois des méthodes qualitatives et quantitatives, les auteurs sont en mesure de fournir une image complète des pratiques et des perspectives actuelles en matière d'ingénierie logicielle écologique. Les données qualitatives permettent de mieux comprendre comment les praticiens envisagent la consommation d'énergie, tandis que les données quantitatives permettent d'obtenir des résultats plus généralisables qui peuvent être utilisés pour éclairer les recherches et les pratiques futures dans ce domaine.

\paragraph{Implémentation de la démarche}
L'implémentation de la démarche se matérialise par l'adoption d'une approche mixte qui permet d'examiner à la fois les pratiques et les perspectives des ingénieurs logiciels actuels en matière d'ingénierie logicielle écologique. La mise en œuvre de cette approche comporte plusieurs étapes :
\begin{enumerate}
    \item Collecte de données qualitatives : les auteurs ont mené des entretiens approfondis avec 18 employés de Microsoft afin de recueillir des données qualitatives sur la manière dont les praticiens pensent à la consommation d'énergie lorsqu'ils développent des logiciels.
    \item Analyse des données qualitatives : ils analysent les données des entretiens à l'aide d'une analyse thématique afin d'identifier des thèmes et des modèles communs dans la manière dont les praticiens abordent la consommation d'énergie.
    \item Développement de l'enquête : en se fondant sur les enseignements tirés de l'analyse des données qualitatives, ils élaborent une enquête ciblée qui est distribuée à 464 praticiens de quatre entreprises (ABB, Google, IBM et Microsoft).
    \item Collecte de données quantitatives : ils recueillent des données quantitatives à partir des réponses à l'enquête, qui comprennent des questions sur les pratiques et les perspectives en matière de consommation d'énergie.
    \item Analyse des données quantitatives : ils utilisent des techniques d'analyse statistique pour analyser les données de l'enquête et identifier les corrélations entre différentes variables, telles que la taille de l'entreprise et la consommation d'énergie.
    \item Intégration des résultats qualitatifs et quantitatifs : ils intègrent les résultats qualitatifs et quantitatifs pour dresser un tableau complet des pratiques et perspectives actuelles en matière d'ingénierie logicielle écologique.
\end{enumerate}
L'utilisation de cette approche mixte permet d'obtenir une compréhension plus approfondie des pratiques d'ingénierie logicielle écologique, en combinant les aspects qualitatifs et quantitatifs, ce qui ne serait pas possible avec seulement l'une de ces méthodes. En combinant les deux types de données, ils sont en mesure de fournir de riches informations sur la façon dont les praticiens pensent à la consommation d'énergie lorsqu'ils développent des logiciels, tout en identifiant des tendances généralisables qui peuvent éclairer la recherche et la pratique futures dans ce domaine.

\paragraph{Les résultats}
Cet article fait état de plusieurs résultats basés sur l'approche mixte utilisée par les auteurs. Voici quelques-uns des principaux résultats :
\begin{enumerate}
    \item La consommation d'énergie n'est pas une priorité pour la plupart des ingénieurs logiciels : les résultats de l'enquête montrent que la consommation d'énergie n'est pas une priorité absolue pour la plupart des ingénieurs en logiciel, puisque seulement 18 \% des répondants ont indiqué qu'ils considéraient la consommation d'énergie comme un facteur très important dans leur travail.
    \item Manque de sensibilisation et de connaissances sur la consommation d'énergie : de nombreux ingénieurs logiciels manquent de sensibilisation et de connaissances sur la consommation d'énergie dans le cadre du développement de logiciels. Par exemple, seul 29 \% des répondants ont déclaré avoir reçu une formation sur les pratiques de programmation économes en énergie.
    \item Obstacles à la mise en œuvre de pratiques éco-énergétiques : les résultats de l'enquête suggèrent qu'il existe plusieurs obstacles à la mise en œuvre de pratiques éco-énergétiques dans le développement de logiciels, notamment le manque de temps, le manque de ressources et le manque de soutien de la part de la direction.
    \item Possibilités d'amélioration : les auteurs identifient plusieurs possibilités d'amélioration de l'efficacité énergétique dans le développement de logiciels, telles que le renforcement de la formation et de l'éducation sur les pratiques de programmation économes en énergie, l'intégration des considérations énergétiques dans le processus de développement de logiciels, et le développement d'outils et de mesures pour mesurer la consommation d'énergie.
\end{enumerate}
L'étude donne un aperçu des pratiques et des perspectives actuelles en matière d'ingénierie logicielle écologique et met en évidence les domaines dans lesquels des améliorations peuvent être apportées pour réduire l'impact environnemental du développement de logiciels.

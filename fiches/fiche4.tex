% Fiche nº4

\chapter{Fiche nº4} % Main appendix title
\label{app:Fiche4} % For referencing this appendix elsewhere, use \ref{app:Fiche4}

%%%%%%%%%%%%%%%%%%%%%%%%%%%%%%%%%%%%%%%%%%%%%%%%%%%%%%%%%%%%%%%%%%%%%%%%%%%%%%%%%%
\section{Description de l'article}

\paragraph{Titre de l'article~: \textnormal{Safety, Security, Now Sustainability: The Nonfunctional Requirement for the 21st Century}}
\paragraph{Lien de l'article~: \textnormal{https://ieeexplore.ieee.org/document/6728940}}
\paragraph{Liste des auteurs~: \textnormal{Birgit Penzenstadler, Ankita Raturi, Debra Richardson,
and Bill Tomlinson}}
\paragraph{Affiliation des auteurs~: \textnormal{University of California, Irvine}}
\paragraph{Nom de la conférence / revue~: \textnormal{IEEE Software}}
\paragraph{Classification de la conférence / revue~: \textnormal{Q1}}
\paragraph{Nombre de citations de l'article (quelle source ?)~: \textnormal{119 citations (Semantic Scholar)}}



%%%%%%%%%%%%%%%%%%%%%%%%%%%%%%%%%%%%%%%%%%%%%%%%%%%%%%%%%%%%%%%%%%%%%%%%%%%%%%%%%%
\section{Synthèse de l'article}

\paragraph{Problématique}
L'article "Safety, Security, Now Sustainability: The Nonfunctional Requirement for the 21st Century" expose l'importance de la durabilité en tant qu'exigence non-fonctionnelle dans le génie logiciel. Les auteurs soutiennent que les ingénieurs en logiciel peuvent contribuer à la durabilité en considérant le deuxième et le troisième ordre des systèmes logiciels.

Le début de l'article met l'accent sur l'omniprésence croissante des systèmes de programmation complexes dans les activités humaines. Bien que le logiciel ait apporté de nombreux avantages, il a également posé des problèmes de sécurité et de sûreté. En outre, les chercheurs et les spécialistes en génie logiciel ont dû tenir compte à la fois de ces deux problèmes lors du développement de grands systèmes logiciels. Toutefois, les auteurs soutiennent que les concepteurs de logiciels doivent également tenir compte de la fiabilité des systèmes logiciels.

Qui plus est, ils définissent la durabilité comme \textit{"la capacité d'un système à maintenir sa fonction dans le temps, tout en minimisant les impacts négatifs sur l'environnement, la société et l'économie"}. Ils affirment que les ingénieurs en logiciel sont capables d'apporter leur contribution à la durabilité en considérant le deuxième et le troisième ordre des systèmes logiciels. Les effets de second ordre font référence aux effets indirects des systèmes logiciels, tels que les effets sur l'environnement ou la société. Tandis que les effets de troisième ordre sont les effets à long terme des systèmes logiciels, tels que les effets sur les générations futures.

Pour finir, ils prétendent que les ingénieurs informaticiens doivent prendre en considération ces effets de seconds et troisièmes ordres lors de la conception de systèmes logiciels. De la même manière, ils suggèrent plusieurs façons, pour les concepteurs de logiciels, d'intégrer la durabilité dans leur processus de conception, comme avoir un impact sur la durabilité ou utiliser un cadre de durabilité. Toujours est-il qu'ils reconnaissent que l'application de la résilience en tant qu'exigence non-fonctionnelle des systèmes logiciels est problématique.

\paragraph{Pistes possibles (pointés par les auteurs)}
Ici, l'article propose plusieurs pistes possibles pour intégrer la durabilité dans les processus de conception. Primo, une évaluation de l'impact sur la durabilité est suggérée, afin d'analyser les conséquences environnementales, sociales et économiques d'un système logiciel sur l'ensemble de son cycle de vie. Secundo, l'utilisation d'un cadre de durabilité est recommandée, offrant des principes et des lignes directrices pour la conception de systèmes logiciels durables. Tertio, en intégrant la durabilité dans le processus d'ingénierie, en identifiant les exigences spécifiques liées à la durabilité et en les incluant dans la spécification des exigences logicielles. Quarto, l'adoption de pratiques de développement de logiciels durables est préconisée, telles que l'utilisation de méthodes agiles et de logiciels libres pour réduire le gaspillage et minimiser l'impact environnemental. Quinto, une attention particulière est portée à l'architecture logicielle et aux modèles de conception, avec la proposition de concevoir des systèmes modulaires, évolutifs et économes en énergie.

\paragraph{Question de recherche}
\begin{itemize}
    \item Quelles sont les meilleures pratiques actuelles pour concevoir des systèmes logiciels sûrs et sécurisés ?
    \item Comment les ingénieurs en logiciel peuvent-ils équilibrer efficacement les exigences vis à vis de la sécurité et de la sûreté, avec d'autres exigences non-fonctionnelles, telles que les performances ou la facilité d'utilisation ?
    \item Quelles sont les méthodes les plus efficaces pour tester et évaluer la sécurité et la sûreté des systèmes logiciels ?
    \item Quelles sont les considérations éthiques liées à la sécurité et à la sûreté des logiciels ?
\end{itemize}

\paragraph{Démarche adoptée}
Les auteurs de l'article adoptent une approche d'analyse documentaire pour explorer l'importance de la durabilité en tant qu'exigence non fonctionnelle dans l'ingénierie logicielle. Ils passent en revue les recherches existantes sur la durabilité et l'ingénierie logicielle, y compris les recherches sur les impacts de deuxième et troisième ordre des systèmes logiciels, les cadres de durabilité et les pratiques de développement de logiciels durables.

Ils donnent également plusieurs exemples sur la façon dont les systèmes logiciels ont des impacts de deuxième et troisième ordre sur l'environnement, la société et l'économie. Ils suggèrent aux ingénieurs en logiciel plusieurs méthodes afin d'intégrer la durabilité dans leur processus de conception, comme par exemple en réalisant une évaluation de l'impact sur la durabilité ou en utilisant un cadre de durabilité. 

À la fin, ils utilisent une approche qualitative pour explorer le thème de la durabilité en tant qu'exigence non-fonctionnelle dans l'ingénierie logicielle. Ils s'appuient sur les recherches existantes et fournissent des exemples et des suggestions sur la manière dont les ingénieurs en logiciel peuvent intégrer la durabilité dans leur processus de conception.

\paragraph{Implémentation de la démarche}
L'implémentation de l'approche dans cet article implique la réalisation d'une analyse documentaire des recherches existantes sur la durabilité et le génie logiciel. Les auteurs s'appuient sur une série de sources, en particulier des articles universitaires, des livres et des rapports, pour étudier l'importance de la durabilité en tant qu'exigence non-fonctionnelle dans le génie logiciel. Ils analysent la littérature pour identifier les thèmes clés liés à la durabilité et au génie logiciel, tels que les impacts de deuxième et troisième ordre des systèmes logiciels sur l'environnement, la société et l'économie.

Les auteurs donnent des exemples de la manière dont les ingénieurs en logiciel peuvent intégrer la durabilité dans leur processus de conception, tels que en réalisant une évaluation de l'impact sur la durabilité ou en utilisant un cadre de durabilité. Ils distinguent également les défis que pose la mise en œuvre de la durabilité en tant qu'exigence non-fonctionnelle pour les systèmes logiciels, comme le manque de sensibilisation ou de compréhension de la durabilité parmi les ingénieurs en logiciel et les compromis entre la durabilité et d'autres exigences non-fonctionnelles.

Cette mise en œuvre implique un examen approfondi de la recherche existante et l'identification de thèmes clés et de suggestions sur la manière dont les ingénieurs logiciels intègrent la durabilité dans leur processus de conception. L'analyse documentaire est une méthode couramment utilisée dans la recherche afin de synthétiser les connaissances existantes sur un sujet particulier. En s'appuyant sur un éventail de sources, les auteurs s'assurent de fournir une vue d'ensemble de l'importance de la durabilité dans l'ingénierie logicielle et d'identifier les domaines clés pour des recherches plus approfondies.

L'un des points forts de cette approche est qu'elle permet aux chercheurs d'identifier les faiblesses dans la littérature existante et de suggérer des domaines de recherche future, tels que le développement de mesures de durabilité pour les systèmes logiciels et l'intégration de la durabilité dans les méthodologies de développement de logiciels.

\paragraph{Les résultats}
Globalement, le résultat de l'article est un appel à l'action pour les ingénieurs en logiciel afin qu'ils considèrent la durabilité comme une exigence non-fonctionnelle dans leur processus de conception. Les auteurs laissent entendre qu'en intégrant la durabilité dans leur processus de conception, les ingénieurs en logiciel ont la capacité de créer des systèmes logiciels plus responsables sur le plan environnemental et social. L'article met également en évidence les défis liés à la mise en place de la durabilité en tant qu'exigence non-fonctionnelle, tels que le manque de sensibilisation ou de compréhension de la durabilité parmi les ingénieurs en logiciel et les compromis entre la durabilité et d'autres exigences non-fonctionnelles.

Le résultat de l'article n'est ni une solution ni un outil spécifique pour intégrer la durabilité dans l'ingénierie logicielle, mais plutôt un ensemble de thèmes clé et de suggestions pour des recherches ultérieures.

Enfin, le document est une contribution précieuse au domaine de l'ingénierie logicielle et de la durabilité, offrant un aperçu de l'importance de la durabilité en tant qu'exigence non-fonctionnelle et suggérant des domaines de recherche et d'actions ultérieures.

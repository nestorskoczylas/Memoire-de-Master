% Fiche nº5

\chapter{Fiche nº5} % Main appendix title
\label{app:Fiche5} % For referencing this appendix elsewhere, use \ref{app:Fiche5}

%%%%%%%%%%%%%%%%%%%%%%%%%%%%%%%%%%%%%%%%%%%%%%%%%%%%%%%%%%%%%%%%%%%%%%%%%%%%%%%%%%
\section{Description de l'article}

\paragraph{Titre de l'article~: \textnormal{Integrating Sustainability Metrics into Project and Portfolio Performance Assessment in Agile Software Development A Data-Driven Scoring Model}}
\paragraph{Lien de l'article~: \textnormal{https://doi.org/10.3390/su151713139}}
\paragraph{Liste des auteurs~: \textnormal{Cristian Fagarasan, Ciprian Cristea, Maria Cristea, Ovidiu Popa and Adrian Pisla}}
\paragraph{Affiliation des auteurs~: \textnormal{Universities and research institutions in Switzerland}}
\paragraph{Nom de la conférence / revue~: \textnormal{Sustainability}}
\paragraph{Classification de la conférence / revue~: \textnormal{Q1}}
\paragraph{Nombre de citations de l'article (quelle source ?)~: \textnormal{5 citations (Google Scholar)}}



%%%%%%%%%%%%%%%%%%%%%%%%%%%%%%%%%%%%%%%%%%%%%%%%%%%%%%%%%%%%%%%%%%%%%%%%%%%%%%%%%%
\section{Synthèse de l'article}

\paragraph{Problématique}
L'article "Integrating Sustainability Metrics into Project and Portfolio Performance Assessment in Agile Software Development: A Data-Driven Scoring Model" aborde la problématique de l'intégration des métriques de durabilité dans l'évaluation des performances des projets et des portefeuilles dans le contexte du développement logiciel agile. Cette problématique émerge de la reconnaissance croissante de l'importance de la durabilité dans le secteur du développement logiciel, où les impacts sociaux, économiques et environnementaux sont de plus en plus pris en compte. Malgré cette prise de conscience, il existe encore un manque dans la littérature concernant la manière d'intégrer efficacement les métriques de durabilité dans l'évaluation des performances des projets et des portefeuilles dans les environnements de développement logiciel agile.

L'objectif principal de l'article est de combler cette lacune en proposant un modèle de notation basé sur les données qui combine efficacement les principes de gestion de projet avec les considérations de durabilité. Ce modèle vise à permettre la mesure de la durabilité du code, offrant ainsi une compréhension plus approfondie de la relation entre les performances de livraison des projets et les pratiques durables. En mettant en avant l'importance de promouvoir la durabilité à long terme tout au long du processus de développement logiciel, l'article cherche à relever les défis potentiels associés à la livraison de projets dans un contexte agile.

En résumé, la problématique centrale de cet article réside dans la nécessité d'intégrer de manière efficace les métriques de durabilité dans l'évaluation des performances des projets et des portefeuilles dans le développement logiciel agile, afin de favoriser des pratiques plus durables et efficientes tout en répondant aux exigences croissantes en matière de durabilité dans le domaine du développement logiciel.

\paragraph{Pistes possibles (pointés par les auteurs)}
Les auteurs de l'article "Integrating Sustainability Metrics into Project and Portfolio Performance Assessment in Agile Software Development: A Data-Driven Scoring Model" soulignent plusieurs pistes possibles pour aborder la problématique de l'intégration des métriques de durabilité dans l'évaluation des performances des projets et des portefeuilles dans le développement logiciel agile. Parmi ces pistes, on peut citer :
\begin{itemize}
    \item Développement d'un modèle de notation basé sur les données : Les auteurs proposent la création d'un modèle de notation qui intègre des métriques de durabilité spécifiques pour évaluer les performances des projets et des portefeuilles dans un contexte agile. Ce modèle permettrait de mesurer la durabilité du code et d'identifier les pratiques durables qui contribuent à l'efficacité des livraisons de projets.
    \item Analyse des performances de livraison et des pratiques durables : Les auteurs suggèrent d'analyser en profondeur les performances de livraison des projets et de mettre en évidence les pratiques durables qui ont un impact significatif sur ces performances. Cela permettrait de mieux comprendre comment les aspects de durabilité peuvent être intégrés de manière efficace dans les processus de développement logiciel agile.
    \item Validation du modèle à travers des études de cas réelles : Une autre piste explorée est la validation du modèle de notation proposé à travers des études de cas réelles. Cette approche permettrait de tester la robustesse et l'applicabilité du modèle dans des situations concrètes, offrant ainsi des insights pratiques sur son utilisation et son efficacité.
\end{itemize}
En explorant ces différentes pistes, les auteurs visent à fournir aux organisations du secteur du développement logiciel des outils et des méthodes pour évaluer de manière plus holistique les performances de leurs projets et portefeuilles, tout en intégrant des considérations de durabilité essentielles pour répondre aux défis actuels et futurs du domaine.

\paragraph{Questions de recherche}
\begin{itemize}
    \item Quels sont les principaux défis rencontrés par les organisations du secteur du développement logiciel lorsqu'il s'agit d'incorporer des métriques de durabilité dans l'évaluation de leurs performances de projet et de portefeuille ?
    \item Quelles sont les métriques de durabilité les plus pertinentes et significatives à intégrer dans les évaluations de performances des projets et des portefeuilles dans un environnement agile ?
    \item Comment le modèle de notation basé sur les données proposé peut-il être adapté et mis en œuvre dans différentes organisations de développement logiciel pour améliorer à la fois les performances de livraison et la durabilité des pratiques ?
    \item Quel impact concret l'intégration de métriques de durabilité a-t-elle sur la gestion des portefeuilles de projets dans le développement logiciel agile en termes d'efficacité opérationnelle, de rentabilité et de durabilité à long terme ?
    \item Quelles sont les meilleures pratiques et recommandations pour les organisations cherchant à intégrer de manière efficace des considérations de durabilité dans leurs processus de gestion de projet et de portefeuille dans un contexte agile ?
\end{itemize}

\paragraph{Démarche adoptée}
La démarche adoptée dans l'article "Integrating Sustainability Metrics into Project and Portfolio Performance Assessment in Agile Software Development: A Data-Driven Scoring Model" se distingue par sa nature conceptuelle et stratégique, mettant en avant les principes et les objectifs sous-jacents de l'intégration des métriques de durabilité dans les évaluations de performances des projets et des portefeuilles dans le contexte du développement logiciel agile. Voici une analyse de la démarche conceptuelle adoptée dans l'étude :
\begin{enumerate}
    \item Identification des besoins et des enjeux : La démarche a débuté par une analyse approfondie des besoins et des enjeux actuels dans le domaine de la gestion de projet et de portefeuille dans le contexte du développement logiciel agile. Cela inclut la reconnaissance croissante de l'importance de la durabilité et de l'efficacité dans les pratiques de développement logiciel.
    \item Définition des objectifs et des principes directeurs : Sur la base de l'analyse des besoins, les auteurs ont défini des objectifs clairs pour l'intégration des métriques de durabilité dans les évaluations de performances. Les principes directeurs, tels que l'alignement avec les objectifs de durabilité, la transparence et la prise de décision basée sur les données, ont été établis pour guider le développement du modèle.
    \item Conception du cadre conceptuel : Un cadre conceptuel a été élaboré pour définir les concepts clés, les relations et les interactions entre les métriques de durabilité, les performances des projets et des portefeuilles, et les pratiques de développement logiciel agile. Ce cadre a permis de structurer la réflexion autour de l'intégration des métriques de durabilité.
    \item Élaboration du modèle stratégique : En se basant sur le cadre conceptuel, les auteurs ont développé un modèle stratégique qui mettait en avant l'importance de la durabilité comme un pilier essentiel des performances des projets et des portefeuilles dans le développement logiciel agile. Ce modèle visait à promouvoir une approche holistique et intégrée de la durabilité.
    \item Analyse des implications et des perspectives : Les implications stratégiques de l'intégration des métriques de durabilité ont été discutées, mettant en avant les avantages à long terme pour les organisations de développement logiciel. Les perspectives futures ont été explorées pour souligner l'importance de cette démarche conceptuelle dans un contexte en évolution constante.
\end{enumerate}
En adoptant cette démarche conceptuelle et stratégique, les auteurs ont pu proposer un modèle novateur pour intégrer les métriques de durabilité dans les évaluations de performances des projets et des portefeuilles dans le développement logiciel agile, offrant ainsi une vision globale et théorique sur cette problématique.

\paragraph{Implémentation de la démarche}
L'implémentation de la démarche dans l'article "Integrating Sustainability Metrics into Project and Portfolio Performance Assessment in Agile Software Development: A Data-Driven Scoring Model" s'est déroulée de manière structurée et méthodique. Voici comment la démarche a été mise en œuvre :
\begin{enumerate}
    \item Revues de la littérature et identification des lacunes : Les auteurs ont effectué une revue approfondie de la littérature existante pour comprendre les pratiques actuelles en matière d'évaluation des performances des projets et des portefeuilles dans le développement logiciel agile, ainsi que les lacunes en termes d'intégration des métriques de durabilité. Cette étape a permis de poser les bases conceptuelles de leur recherche.
    \item Développement du modèle de notation : En se basant sur les insights de la revue de la littérature, les auteurs ont développé un modèle de notation basé sur les données qui intègre des métriques de durabilité spécifiques. Ce modèle a été conçu pour être adaptable aux besoins des organisations de développement logiciel et pour faciliter l'évaluation des performances et de la durabilité des projets et des portefeuilles.
    \item Validation à travers une étude de cas : Pour tester l'efficacité et la pertinence de leur modèle, les auteurs ont mené une étude de cas pratique. Cette étape a impliqué l'application du modèle de notation dans un contexte réel de développement logiciel agile, en utilisant des données concrètes pour évaluer les performances et la durabilité des projets et des portefeuilles.
    \item Analyse des résultats et implications : Les résultats de l'étude de cas ont été analysés pour évaluer l'impact du modèle de notation sur les performances et la durabilité dans le développement logiciel agile. Les auteurs ont identifié les avantages et les limites de leur approche, et ont discuté des implications pratiques pour les organisations cherchant à intégrer des métriques de durabilité dans leurs évaluations de performances.
\end{enumerate}
En mettant en œuvre cette démarche itérative et en validant leur modèle à travers une étude de cas, les auteurs ont pu démontrer la faisabilité et l'efficacité de leur approche pour intégrer des métriques de durabilité dans l'évaluation des performances des projets et des portefeuilles dans le développement logiciel agile.

\paragraph{Les résultats}
Les résultats de l'étude "Integrating Sustainability Metrics into Project and Portfolio Performance Assessment in Agile Software Development: A Data-Driven Scoring Model" mettent en lumière plusieurs aspects importants liés à l'intégration des métriques de durabilité dans les évaluations de performances des projets et des portefeuilles dans le contexte du développement logiciel agile.

Les résultats de l'étude ont démontré l'efficacité du modèle de notation proposé pour intégrer les métriques de durabilité dans les évaluations de performances. Le modèle a permis aux organisations de mesurer et d'améliorer la durabilité de leurs pratiques de développement logiciel tout en maintenant un focus sur la performance des projets et des portefeuilles.

L'étude a révélé que l'intégration des métriques de durabilité dans les évaluations de performances avait un impact significatif sur la performance globale des projets et des portefeuilles. En mettant l'accent sur la durabilité, les organisations ont pu améliorer leur efficacité opérationnelle tout en réduisant leur empreinte environnementale.

Grâce au modèle de notation basé sur les données, les organisations ont pu prendre des décisions plus informées en matière de gestion de projet et de portefeuille. Les métriques de durabilité ont fourni des indicateurs clés pour évaluer l'impact environnemental et social des pratiques de développement logiciel.

L'étude de cas menée pour valider le modèle de notation a confirmé sa pertinence et son applicabilité dans un environnement réel de développement logiciel agile. Les résultats de l'étude de cas ont illustré comment le modèle pouvait être mis en œuvre avec succès pour améliorer la durabilité et la performance des projets et des portefeuilles.

Les résultats de l'étude ont apporté des contributions significatives à la littérature académique en matière d'intégration des métriques de durabilité dans les évaluations de performances. De plus, les pratiques industrielles ont bénéficié des insights et des recommandations issus de cette recherche pour améliorer leurs processus de développement logiciel.
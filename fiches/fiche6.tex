% Fiche nº6

\chapter{Fiche nº6} % Main appendix title
\label{app:Fiche6} % For referencing this appendix elsewhere, use \ref{app:Fiche6}

%%%%%%%%%%%%%%%%%%%%%%%%%%%%%%%%%%%%%%%%%%%%%%%%%%%%%%%%%%%%%%%%%%%%%%%%%%%%%%%%%%
\section{Description de l'article}

\paragraph{Titre de l'article~: \textnormal{Green and Sustainable Software Engineering - a Systematic Mapping Study}}
\paragraph{Lien de l'article~: \textnormal{https://dl.acm.org/doi/10.1145/3275245.3275258}}
\paragraph{Liste des auteurs~: \textnormal{Brunna C. Mourão, Leila Karita, Ivan do Carmo Machado}}
\paragraph{Nom de la conférence / revue~: \textnormal{17th Brazilian Symposium on Software Quality (SBQS)}}
\paragraph{Classification de la conférence / revue~: \textnormal{C}}
\paragraph{Nombre de citations de l'article (quelle source ?)~: \textnormal{25 citations (Google Scholar)}}



%%%%%%%%%%%%%%%%%%%%%%%%%%%%%%%%%%%%%%%%%%%%%%%%%%%%%%%%%%%%%%%%%%%%%%%%%%%%%%%%%%
\section{Synthèse de l'article}

\paragraph{Problématique}
La problématique de l'article "Green and Sustainable Software Engineering - a Systematic Mapping Study" réside dans la nécessité de comprendre et d'évaluer l'impact de la durabilité dans le domaine de l'ingénierie logicielle. Face à une prise de conscience croissante des enjeux environnementaux et sociaux, il est devenu impératif pour la communauté du génie logiciel d'explorer des pratiques plus durables et respectueuses de l'environnement. Cette étude systématique vise à répondre à des questions clés telles que les types de contributions de l'ingénierie logicielle examinés dans le contexte de la durabilité, les phases du cycle de vie du développement logiciel dédiées aux efforts de durabilité, et les tendances émergentes dans ce domaine en pleine expansion.

En examinant 75 études primaires pertinentes, l'article cherche à identifier les approches, les modèles et les outils proposés pour soutenir le développement de logiciels durables. La problématique centrale réside dans la recherche de solutions innovantes et efficaces pour intégrer des pratiques durables tout au long du cycle de vie du logiciel, de la conception à la maintenance. En mettant en lumière les efforts de la communauté de recherche en génie logiciel pour incorporer des pratiques durables, cette étude vise à identifier les lacunes existantes, à proposer des pistes de recherche futures et à encourager l'adoption généralisée de pratiques respectueuses de l'environnement dans le domaine du logiciel.

En résumé, la problématique de cet article réside dans la nécessité de combler le fossé entre l'ingénierie logicielle traditionnelle et les pratiques durables, en explorant les moyens par lesquels le secteur du logiciel peut contribuer de manière significative à la durabilité environnementale et sociale.

\paragraph{Pistes possibles (pointés par les auteurs)}
Les auteurs de l'article "Green and Sustainable Software Engineering - a Systematic Mapping Study" soulignent plusieurs pistes possibles pour la recherche future dans le domaine de l'ingénierie logicielle verte et durable.

Exploration de techniques, outils et métriques couvrant les phases de construction, de test et de maintenance du logiciel : Les auteurs notent un besoin accru d'études approfondies sur les pratiques durables dans ces phases spécifiques du cycle de vie du logiciel, suggérant que des recherches supplémentaires sont nécessaires pour développer des approches spécifiques et des outils adaptés à ces étapes cruciales du développement logiciel.

Meilleure alignement entre la recherche et la pratique : Il est souligné qu'il existe un besoin clair d'améliorer l'alignement entre la recherche académique et les pratiques industrielles dans le domaine de l'ingénierie logicielle durable. Les auteurs encouragent une collaboration plus étroite entre les chercheurs et les praticiens pour garantir que les avancées de la recherche se traduisent efficacement dans des applications concrètes et bénéfiques pour l'industrie du logiciel.

Développement de cadres, approches et modèles : Les chercheurs mettent en avant l'importance de continuer à proposer des cadres conceptuels, des approches méthodologiques et des modèles pour guider la conception et le développement de logiciels durables. Cette piste de recherche suggère que la création de lignes directrices et de normes spécifiques peut contribuer à une adoption plus large de pratiques durables dans le secteur du génie logiciel.

En explorant ces pistes potentielles, les auteurs espèrent stimuler davantage de recherches et d'innovations dans le domaine de l'ingénierie logicielle verte et durable, contribuant ainsi à une transition vers des pratiques plus durables et respectueuses de l'environnement dans l'industrie du logiciel.

\paragraph{Questions de recherche}
Les questions de recherche identifiées dans l'article "Green and Sustainable Software Engineering - a Systematic Mapping Study" sont les suivantes :

\begin{itemize}
  \item Quels types de contributions de l'ingénierie logicielle ont été étudiés dans le domaine de l'ingénierie logicielle verte et durable?
  \item Si oui, quelles phases du cycle de vie du développement logiciel ont été appliquées pour les efforts de durabilité?
  \item Quels types de preuves ont été utilisés dans les études primaires examinées?
  \item Quels types de recherche ont été menés dans le domaine de l'ingénierie logicielle verte et durable?
  \item Quels domaines d'application ont été abordés par la communauté de recherche, tels que Mobile, Cloud, IoT, Embedded Systems, etc.?
  \item Quels sont les types de validation des méthodes proposées dans les études primaires?
  \item Quels sont les principaux résultats et tendances identifiés dans les études primaires examinées?
\end{itemize}

Ces questions de recherche visent à explorer divers aspects de l'ingénierie logicielle verte et durable, allant des types de contributions étudiés aux domaines d'application spécifiques et aux méthodes de validation utilisées dans les études primaires.

\paragraph{Démarche adoptée}
La démarche adoptée dans l'article "Green and Sustainable Software Engineering - a Systematic Mapping Study" peut être résumée comme suit :
\begin{itemize}
  \item Identification des études pertinentes
  \item Extraction des données
  \item Classification des études
  \item Analyse des résultats
\end{itemize}
Cette démarche méthodologique a permis aux auteurs de systématiquement examiner un ensemble d'études primaires pour répondre aux questions de recherche formulées dans le cadre de l'étude.

\paragraph{Implémentation de la démarche}
L'implémentation de la démarche dans l'article "Green and Sustainable Software Engineering - a Systematic Mapping Study" s'est déroulée en suivant les étapes détaillées suivantes :

\begin{enumerate}
  \item Conception d'un protocole de recherche basé sur les directives de Kitchenham Barbara et de C Stuart.
  \item Définition d'une question de recherche principale, divisée en huit sous-questions.
  \item Collecte de données à partir de six bases de données numériques, avec l'application de critères d'inclusion et d'exclusion pour sélectionner les études primaires pertinentes.
  \item Sélection des articles pertinents en fonction des critères définis, suivi d'une classification des études extraites.
  \item Analyse des données extraites à l'aide d'une approche systématique pour répondre aux questions de recherche formulées.
\end{enumerate}

Cette approche méthodique a permis aux chercheurs d'obtenir une vue d'ensemble complète des études primaires dans le domaine de l'ingénierie logicielle verte et durable, en identifiant les tendances, les lacunes et les opportunités de recherche futures.

\paragraph{Les résultats}
Les résultats de l'article "Green and Sustainable Software Engineering - a Systematic Mapping Study" incluent l'identification de 9 types de contributions différents dans le domaine du Génie Logiciel Vert, tels que les approches, les cadres de travail, les modèles, les techniques, les méthodes, les outils, les lignes directrices, les métriques et les catalogues.

De plus, il analyse de la répartition des publications par type de contribution, montrant une concentration significative sur l'exploration des approches, des cadres de travail et des modèles.

L'évolution du nombre de publications dans le domaine du Génie Logiciel Vert au fil des ans, avec une augmentation significative à partir de 2013, indiquant un intérêt croissant de la communauté de recherche pour ce domaine.

Ces résultats mettent en lumière l'importance croissante accordée au développement de logiciels durables et à l'intégration de concepts de durabilité dans les pratiques de génie logiciel.
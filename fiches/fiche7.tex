% Fiche nº7

\chapter{Fiche nº7} % Main appendix title
\label{app:Fiche7} % For referencing this appendix elsewhere, use \ref{app:Fiche7}

%%%%%%%%%%%%%%%%%%%%%%%%%%%%%%%%%%%%%%%%%%%%%%%%%%%%%%%%%%%%%%%%%%%%%%%%%%%%%%%%%%
\section{Description de l'article}

\paragraph{Titre de l'article~: \textnormal{Sustainability is Stratified: Toward a Better Theory of Sustainable Software Engineering}}
\paragraph{Lien de l'article~: \textnormal{https://dl.acm.org/doi/abs/10.1109/ICSE48619.2023.00169}}
\paragraph{Liste des auteurs~: \textnormal{Sean McGuire, Erin Schultz, Bimpe Ayoola, Paul Ralph}}
\paragraph{Affiliation des auteurs~: \textnormal{Faculty of Computer Science, Dalhousie University, Halifax, NS, Canada}}
\paragraph{Nom de la conférence / revue~: \textnormal{ICSE 2023 (International Conference on Software Engineering)}}
\paragraph{Classification de la conférence / revue~: \textnormal{A}}
\paragraph{Nombre de citations de l'article (quelle source ?)~: \textnormal{10 citations (Google Scholar)}}



%%%%%%%%%%%%%%%%%%%%%%%%%%%%%%%%%%%%%%%%%%%%%%%%%%%%%%%%%%%%%%%%%%%%%%%%%%%%%%%%%%
\section{Synthèse de l'article}

\paragraph{Problématique}
L'article "Sustainability is Stratified: Toward a Better Theory of Sustainable Software Engineering" soulève plusieurs problèmes importants dans le domaine de l'ingénierie logicielle durable (SSE).

Premièrement, les auteurs soulignent que le concept de durabilité est souvent mal défini et appliqué de manière incohérente dans la littérature. Cela rend difficile la comparaison des différentes approches de SSE et l'évaluation de leur efficacité.

Deuxièmement, les études sur la SSE se concentrent principalement sur les aspects environnementaux de la durabilité, tels que la consommation d'énergie et l'empreinte carbone. Les aspects sociaux et économiques de la durabilité, tels que l'impact du logiciel sur les travailleurs et les communautés, sont souvent négligés.

Troisièmement, la plupart des recherches sur la SSE sont théoriques ou conceptuelles. Il existe un manque d'évaluations empiriques de l'efficacité des différentes interventions de SSE.

Quatrièmement, les auteurs affirment que la durabilité est un concept stratifié, c'est-à-dire qu'elle a des significations différentes à différents niveaux d'abstraction (code, logiciel, processus de développement, etc.). Cela rend difficile la prise en compte de la durabilité dans l'ensemble du cycle de vie du logiciel.

Cinquièmement, la durabilité est un concept multisystémique, c'est-à-dire qu'elle émerge de l'interaction entre différents systèmes sociaux, techniques et sociotechniques. Cela rend difficile l'identification des leviers les plus efficaces pour promouvoir la durabilité dans l'ingénierie logicielle.

En conclusion, l'article appelle à une meilleure théorie de la SSE qui tienne compte de la nature stratifiée et multisystémique de la durabilité. Cette théorie devrait être basée sur des recherches empiriques rigoureuses et prendre en compte les aspects environnementaux, sociaux et économiques de la durabilité.

\paragraph{Pistes possibles (pointés par les auteurs)}
Les auteurs de l'article "Sustainability is Stratified: Toward a Better Theory of Sustainable Software Engineering" proposent plusieurs pistes pour une meilleure théorie de la SSE:
\begin{itemize}
    \item Clarifier le concept de durabilité:
        \item Définir les différentes dimensions de la durabilité (environnementale, sociale, économique) dans le contexte de l'ingénierie logicielle.
        \item Développer un cadre conceptuel pour la SSE qui intègre ces différentes dimensions.
    \item Élargir le champ d'investigation de la SSE:
        \item Accorder plus d'attention aux aspects sociaux et économiques de la durabilité.
        \item Étudier l'impact du logiciel sur les travailleurs, les communautés et la société en général.
    \item Renforcer l'aspect empirique de la recherche en SSE:
        \item Réaliser plus d'évaluations empiriques de l'efficacité des différentes interventions de SSE.
        \item Développer des méthodes et des outils pour mesurer la durabilité des logiciels.
    \item Prendre en compte la nature stratifiée de la durabilité:
        \item Développer des approches de SSE adaptées aux différents niveaux d'abstraction (code, logiciel, processus de développement, etc.).
        \item Étudier les interactions entre les différents niveaux de la durabilité.
    \item Considérer la nature multisystémique de la durabilité:
        \item Identifier les différents systèmes qui contribuent à la durabilité du logiciel.
        \item Étudier les interactions entre ces différents systèmes.
    \item Développer des collaborations interdisciplinaires:
        \item Collaborer avec des chercheurs d'autres disciplines, tels que les sciences sociales et environnementales.
        \item Développer des approches de SSE qui intègrent les connaissances et les perspectives de différentes disciplines.
\end{itemize}

\paragraph{Questions de recherche}
\begin{enumerate}
    \item Comment définir et mesurer la durabilité d'un logiciel ?
    \item Quels sont les impacts environnementaux, sociaux et économiques du logiciel ?
    \item Quelles sont les meilleures pratiques pour mettre en œuvre la SSE dans les projets logiciels ?
    \item Quelles politiques et incitations peuvent être mises en place pour encourager l'adoption de la SSE ?
\end{enumerate}

\paragraph{Démarche adoptée}
Les chercheurs ont adopté une démarche méthodique et rigoureuse pour aborder la problématique de l'ingénierie logicielle durable. Leur approche s'est appuyée sur une revue de la littérature existante dans le domaine de la durabilité en ingénierie logicielle, en mettant l'accent sur les dimensions environnementales, sociales, économiques et techniques de la durabilité. Cette revue leur a permis de cerner les lacunes et les défis actuels dans la compréhension de la durabilité dans le contexte du développement logiciel.

En outre, les chercheurs ont mené une synthèse des recherches qualitatives pour approfondir leur compréhension des différents aspects de la durabilité en ingénierie logicielle. Cette approche leur a permis de dégager des tendances, des modèles et des perspectives nouvelles sur la durabilité dans ce domaine, en mettant en lumière la complexité et la diversité des enjeux à considérer.

En combinant ces différentes étapes de recherche, les chercheurs ont pu proposer une théorie plus élaborée de l'ingénierie logicielle durable, mettant en avant la nature stratifiée et multisystémique de la durabilité. Leur démarche a ainsi permis de contribuer de manière significative à la réflexion et à l'avancement des connaissances dans ce domaine émergent et crucial pour l'avenir de l'industrie du logiciel.

\paragraph{Implémentation de la démarche}
Les chercheurs ont mis en œuvre leur démarche en deux phases distinctes. Tout d'abord, ils ont réalisé une revue de la littérature de type "scoping review" pour explorer l'état actuel de la recherche en ingénierie logicielle durable. Cette approche leur a permis d'identifier les lacunes et les opportunités de recherche dans le domaine, ainsi que de déterminer le type d'analyse systématique le plus approprié à entreprendre par la suite. Après avoir opté pour une approche de méta-synthèse, qui consiste en une synthèse qualitative des études primaires, les chercheurs ont sélectionné un sous-ensemble d'études pour générer une théorie améliorée de l'ingénierie logicielle durable.

La méta-synthèse a été réalisée en analysant en profondeur les données qualitatives extraites des études primaires sélectionnées. Cette approche qualitative a permis aux chercheurs de dégager des thèmes récurrents, des modèles émergents et des perspectives nouvelles sur la durabilité en ingénierie logicielle. En combinant les résultats de cette méta-synthèse avec les propositions issues de la revue de littérature initiale, les chercheurs ont pu formuler un modèle novateur de l'ingénierie logicielle durable, mettant en lumière la nature stratifiée et multisystémique de la durabilité dans ce domaine.

En intégrant ces différentes étapes de recherche, de la revue de la littérature à la méta-synthèse, les chercheurs ont pu développer une théorie plus approfondie et nuancée de l'ingénierie logicielle durable. Cette démarche méthodique et rigoureuse a permis aux chercheurs de contribuer de manière significative à la compréhension et à l'avancement des connaissances dans le domaine de la durabilité en ingénierie logicielle.

\paragraph{Les résultats}
Les résultats de l'étude ont permis de mettre en lumière plusieurs aspects importants concernant la durabilité en ingénierie logicielle. Tout d'abord, les chercheurs ont souligné que la durabilité est un concept multisystémique et stratifié, impliquant des interactions complexes entre différents niveaux de systèmes sociaux, techniques et sociotechniques. Cette perspective élargie de la durabilité met en évidence la nécessité de prendre en compte la diversité des dimensions environnementales, sociales et économiques dans le développement de logiciels durables.

De plus, l'analyse des recherches existantes a révélé un manque de recherches empiriques rigoureuses dans le domaine de l'ingénierie logicielle durable, avec une prédominance d'études non empiriques telles que des articles de position. Les chercheurs ont souligné l'importance d'entreprendre davantage d'évaluations empiriques d'interventions visant à améliorer la durabilité des processus et des produits logiciels.

Enfin, les résultats ont mis en évidence une focalisation plus marquée sur la durabilité des produits logiciels par rapport aux processus de développement, ainsi qu'une prépondérance de la durabilité écologique par rapport aux dimensions économiques et sociales de la durabilité. Les chercheurs ont donc appelé à une diversification des recherches pour inclure une gamme plus large de dimensions de durabilité et à développer des instruments plus sophistiqués pour évaluer ces différentes dimensions à différents niveaux de stratification.

En résumé, les résultats de l'étude soulignent la nécessité d'une approche plus holistique et intégrée de la durabilité en ingénierie logicielle, mettant en avant la complexité et la diversité des enjeux à considérer pour promouvoir des pratiques de développement logiciel plus durables et responsables.
% Fiche nº8

\chapter{Fiche nº8} % Main appendix title
\label{app:Fiche8} % For referencing this appendix elsewhere, use \ref{app:Fiche8}

%%%%%%%%%%%%%%%%%%%%%%%%%%%%%%%%%%%%%%%%%%%%%%%%%%%%%%%%%%%%%%%%%%%%%%%%%%%%%%%%%%
\section{Description de l'article}

\paragraph{Titre de l'article~: \textnormal{The Green Software Measurement Structure Based on Sustainability Perspective}}
\paragraph{Lien de l'article~: \textnormal{https://ieeexplore.ieee.org/document/9611108}}
\paragraph{Liste des auteurs~: \textnormal{Komeil Raisian, Jamaiah Yahaya, Siti Rohana Ahmad Ibrahim, Aziz Deraman, Tumen Yunos}}
\paragraph{Nom de la conférence / revue~: \textnormal{2021 International Conference on Electrical Engineering and Informatics (ICEEI)}}
\paragraph{Classification de la conférence / revue~: \textnormal{C}}
\paragraph{Nombre de citations de l'article (quelle source ?)~: \textnormal{INCONNU}}



%%%%%%%%%%%%%%%%%%%%%%%%%%%%%%%%%%%%%%%%%%%%%%%%%%%%%%%%%%%%%%%%%%%%%%%%%%%%%%%%%%
\section{Synthèse de l'article}

\paragraph{Problématique}
La problématique de l'article "The Green Software Measurement Structure Based on Sustainability Perspective" réside dans la nécessité croissante de prendre en compte la durabilité et la "vertitude" des logiciels dans le domaine de l'ingénierie logicielle. Alors que la durabilité est devenue un enjeu majeur dans de nombreux secteurs, y compris celui des technologies de l'information, l'application de ces principes au développement de logiciels reste relativement peu explorée. Ainsi, la question centrale abordée par l'article est de savoir comment intégrer efficacement des mesures de durabilité et des critères "verts" dans le processus de développement de logiciels pour répondre aux exigences croissantes de durabilité de la société et de l'industrie.

De plus, l'article met en lumière le manque de sensibilisation et d'attention portée aux mesures de durabilité dans la communauté de l'ingénierie logicielle. Cette lacune soulève la problématique de l'importance de sensibiliser les acteurs de ce domaine aux enjeux de durabilité et de les inciter à intégrer activement des critères verts dans leurs pratiques de développement. Par conséquent, l'article vise à combler ce fossé en proposant une structure de mesure spécifique pour évaluer la durabilité des logiciels, offrant ainsi une base solide pour une approche plus verte et plus durable du développement logiciel.

Enfin, l'article soulève la question de l'impact potentiel d'une telle approche sur les industries et la société dans son ensemble. En intégrant des mesures de durabilité dans le processus de développement de logiciels, il est possible de créer des produits logiciels plus efficaces, économes en énergie et respectueux de l'environnement. Ainsi, la problématique sous-jacente est de savoir comment cette approche peut contribuer à la transition vers une économie plus verte et à la promotion de pratiques durables dans le domaine de l'ingénierie logicielle.

\paragraph{Pistes possibles (pointés par les auteurs)}
Les auteurs de l'article "The Green Software Measurement Structure Based on Sustainability Perspective" soulignent plusieurs pistes possibles pour aborder la problématique de la durabilité des logiciels et de l'intégration de critères verts dans le développement logiciel :
\begin{itemize}
    \item Développement d'une structure de mesure verte : Les auteurs proposent la création d'une structure de mesure spécifique pour évaluer la durabilité des logiciels. Cette approche permettrait d'introduire des critères de durabilité dans le processus de développement et d'assurer que les logiciels produits répondent aux normes de durabilité et d'efficacité énergétique.
    \item Intégration des dimensions de durabilité : Les auteurs mettent en avant l'importance d'intégrer les trois éléments clés de durabilité - social, économique et environnemental - dans l'évaluation des logiciels. En prenant en compte ces dimensions, il est possible de concevoir des logiciels qui non seulement répondent aux besoins des utilisateurs, mais qui contribuent également à la réduction de l'empreinte environnementale et à la promotion de pratiques durables.
    \item Mesures équilibrées et complètes : Les auteurs soulignent la nécessité de développer des mesures équilibrées et complètes pour évaluer la durabilité des logiciels. En s'appuyant sur des éléments tels que l'efficacité énergétique, l'efficacité des ressources, la convivialité, la productivité, la réduction des coûts, le soutien des employés et le support des outils, il est possible de garantir que les logiciels produits sont véritablement "verts" et durables.
\end{itemize}
En résumé, les pistes proposées par les auteurs visent à établir une approche structurée et complète pour mesurer la durabilité des logiciels, en intégrant des critères verts et en favorisant une transition vers des pratiques de développement logiciel plus durables et respectueuses de l'environnement.

\paragraph{Questions de recherche}
Les questions de recherche abordées dans l'article "The Green Software Measurement Structure Based on Sustainability Perspective" sont les suivantes :
\begin{enumerate}
    \item Comment mesurer la durabilité des logiciels et intégrer des critères verts dans le processus de développement pour garantir des produits logiciels plus respectueux de l'environnement et durables ?
    \item Quels sont les éléments clés de durabilité à prendre en compte dans l'évaluation des logiciels, notamment du point de vue social, économique et environnemental ?
    \item Comment développer des mesures équilibrées et complètes pour évaluer la durabilité des logiciels, en prenant en considération des aspects tels que l'efficacité énergétique, l'efficacité des ressources, la convivialité, la productivité, la réduction des coûts, le soutien des employés et le support des outils ?
    \item Quel impact une approche de développement logiciel plus durable et respectueuse de l'environnement pourrait-elle avoir sur les industries et la société dans son ensemble ?
\end{enumerate}
En se concentrant sur ces questions de recherche, les auteurs cherchent à explorer et à proposer des solutions pour intégrer efficacement des mesures de durabilité et des critères verts dans le domaine de l'ingénierie logicielle, en vue de promouvoir des pratiques de développement plus durables et respectueuses de l'environnement.

\paragraph{Démarche adoptée}
Dans l'article "The Green Software Measurement Structure Based on Sustainability Perspective", les auteurs ont suivi une démarche méthodique pour aborder la question de la durabilité des logiciels. Ils ont commencé par effectuer des revues théoriques pour explorer les concepts de durabilité des logiciels et les pratiques de développement durable dans le domaine de l'ingénierie logicielle. Ensuite, ils ont développé une structure de mesure verte, comprenant des éléments clés de durabilité, des mesures associées, des sous-mesures détaillées et des métriques pour évaluer la performance environnementale, sociale et économique des logiciels. Les auteurs ont identifié les dimensions essentielles de durabilité, telles que l'efficacité énergétique, l'efficacité des ressources, la convivialité, la productivité, la réduction des coûts, le soutien des employés et le support des outils, et les ont décomposées en mesures spécifiques pour évaluer la durabilité des logiciels. Cette approche structurée et détaillée vise à promouvoir le développement de logiciels plus durables et respectueux de l'environnement, en intégrant des critères verts et en favorisant une transition vers des pratiques de développement logiciel plus durables.

\paragraph{Implémentation de la démarche}
Dans l'article "The Green Software Measurement Structure Based on Sustainability Perspective", la démarche est implémentée de la manière suivante :
\begin{itemize}
    \item Les auteurs ont effectué des revues théoriques pour explorer les concepts de durabilité des logiciels, les mesures de durabilité et les pratiques de développement durable dans le domaine de l'ingénierie logicielle. Ces revues ont permis de consolider les connaissances existantes et de mettre en évidence les lacunes dans le domaine de la durabilité des logiciels.
    \item Sur la base des revues théoriques, les auteurs ont proposé une structure de mesure spécifique pour évaluer la durabilité des logiciels. Cette structure comprend des éléments clés de durabilité, des mesures associées, des sous-mesures détaillées et des métriques pour évaluer la performance environnementale, sociale et économique des logiciels.
    \item Les auteurs ont identifié les éléments clés de durabilité à prendre en compte dans l'évaluation des logiciels, tels que l'efficacité énergétique, l'efficacité des ressources, la convivialité, la productivité, la réduction des coûts, le soutien des employés et le support des outils. Ces éléments ont été décomposés en mesures spécifiques pour évaluer la durabilité des logiciels.
    \item Les auteurs ont décomposé les éléments de durabilité en mesures et sous-mesures détaillées, permettant une évaluation approfondie de la durabilité des logiciels. Cette approche structurée a permis d'identifier des critères précis pour évaluer différents aspects de la durabilité des logiciels, en intégrant des dimensions environnementales, sociales et économiques.
\end{itemize}
En mettant en œuvre cette démarche, les auteurs ont pu proposer une approche systématique et détaillée pour mesurer la durabilité des logiciels, en intégrant des critères verts et en favorisant le développement de logiciels plus durables et respectueux de l'environnement.

\paragraph{Les résultats}
Les résultats de l'article "The Green Software Measurement Structure Based on Sustainability Perspective" incluent :
\begin{enumerate}
    \item Proposition d'une structure de mesure verte : Les auteurs ont proposé une structure de mesure spécifique pour évaluer la durabilité des logiciels, en intégrant des éléments clés de durabilité, des mesures associées, des sous-mesures détaillées et des métriques pour évaluer la performance environnementale, sociale et économique des logiciels.
    \item Identification des éléments clés de durabilité : Les auteurs ont identifié les dimensions essentielles de durabilité à prendre en compte dans l'évaluation des logiciels, telles que l'efficacité énergétique, l'efficacité des ressources, la convivialité, la productivité, la réduction des coûts, le soutien des employés et le support des outils.
    \item Décomposition des éléments en mesures et sous-mesures : Les éléments de durabilité ont été décomposés en mesures et sous-mesures spécifiques, permettant une évaluation détaillée de la durabilité des logiciels. Cette approche structurée fournit des critères précis pour évaluer différents aspects de la durabilité des logiciels, en intégrant des dimensions environnementales, sociales et économiques.
\end{enumerate}
En résumé, les résultats de l'article fournissent une approche systématique et détaillée pour mesurer la durabilité des logiciels, en mettant l'accent sur des critères verts et en favorisant le développement de logiciels plus durables et respectueux de l'environnement.
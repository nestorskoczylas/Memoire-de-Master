% Fiche nº9

\chapter{Fiche nº9} % Main appendix title
\label{app:Fiche9} % For referencing this appendix elsewhere, use \ref{app:Fiche}

%%%%%%%%%%%%%%%%%%%%%%%%%%%%%%%%%%%%%%%%%%%%%%%%%%%%%%%%%%%%%%%%%%%%%%%%%%%%%%%%%%
\section{Description de l'article}

\paragraph{Titre de l'article~: \textnormal{Sustainable software engineering - have we neglected the software engineer's perspective?}}
\paragraph{Lien de l'article~: \textnormal{https://ieeexplore.ieee.org/document/9679832}}
\paragraph{Liste des auteurs~: \textnormal{Binish Tanveer}}
\paragraph{Affiliation des auteurs~: \textnormal{Department of Software Engineering, Blekinge Institute of Technology, Karlskrona, Sweden}}
\paragraph{Nom de la conférence / revue~: \textnormal{2021 36th IEEE/ACM International Conference on Automated Software Engineering Workshops (ASEW)}}
\paragraph{Classification de la conférence / revue~: \textnormal{Rank A*}}
\paragraph{Nombre de citations de l'article (quelle source ?)~: \textnormal{3 citations (IEEE)}}



%%%%%%%%%%%%%%%%%%%%%%%%%%%%%%%%%%%%%%%%%%%%%%%%%%%%%%%%%%%%%%%%%%%%%%%%%%%%%%%%%%
\section{Synthèse de l'article}

\paragraph{Problématique}
L'article "Sustainable software engineering - have we neglected the software engineer's perspective" de Binish Tanveer soulève des points importants concernant la nécessité de prendre en compte la dimension humaine de la durabilité dans l'ingénierie logicielle. Bien que l'auteur souligne l'importance de comprendre et d'améliorer la durabilité de l'ingénieur pour un développement logiciel de haute qualité, il y a plusieurs domaines dans lesquels l'article pourrait être développé ou amélioré.

L'un des principaux problèmes de l'article est le manque d'exploration et d'analyse approfondies des défis spécifiques auxquels sont confrontés les ingénieurs en logiciel en termes de durabilité. Bien que l'auteur reconnaisse l'importance de facteurs tels que la charge cognitive, les biais cognitifs et l'épuisement professionnel dans le développement de logiciels, une discussion plus détaillée et des preuves empiriques sont nécessaires pour étayer ces affirmations. Des exemples concrets, des études de cas ou des données provenant d'études pertinentes pourraient renforcer l'argumentation et rendre l'article plus convaincant.

En outre, l'article pourrait bénéficier d'un examen plus complet de la littérature existante sur le thème de la durabilité de l'ingénieur dans le génie logiciel. Bien que l'auteur fasse référence à des études secondaires et à des domaines de recherche liés à la durabilité, à la motivation, aux facteurs humains et à la cognition dans le génie logiciel, un examen plus approfondi de ces sources et de leurs implications pour le domaine améliorerait la profondeur et la crédibilité de l'article.

Par ailleurs, l'article manque d'un cadre ou d'une méthodologie claire pour atteindre les objectifs de recherche décrits dans le document. Bien que les objectifs de la recherche soient articulés, comme l'analyse du processus de développement de logiciels du point de vue de la durabilité de l'ingénieur et l'identification de mesures d'amélioration, il est nécessaire d'avoir une approche structurée ou un plan de recherche pour atteindre ces objectifs de manière efficace. La fourniture d'une méthodologie ou d'un cadre de recherche détaillé ajouterait de la clarté et de la rigueur à l'étude.

En conclusion, bien que l'article "Sustainable software engineering - have we neglected the software engineer's perspective" de Binish Tanveer aborde un sujet important, il existe des possibilités de développement en termes de profondeur d'analyse, de revue de la littérature, de méthodologie de recherche et d'intégration des différentes dimensions de la durabilité dans l'ingénierie logicielle. En développant ces aspects, l'article pourrait apporter une contribution plus complète et plus percutante au domaine.

\paragraph{Pistes possibles (pointés par les auteurs)}
Les auteurs de l'article "Sustainable software engineering - have we neglected the software engineer's perspective" (Ingénierie logicielle durable - avons-nous négligé le point de vue de l'ingénieur logiciel) proposent plusieurs pistes de recherche et d'amélioration dans le domaine de l'ingénierie logicielle durable du point de vue de l'ingénieur :

\begin{enumerate}
    \item Explorer la dimension individuelle (humaine) : Les auteurs soulignent la nécessité d'étudier la dimension individuelle (humaine) de la durabilité dans le développement de logiciels. En se concentrant sur les facteurs qui ont un impact sur le bien-être, la santé mentale et les besoins professionnels des ingénieurs en logiciel, les recherches futures pourront mieux comprendre et relever les défis auxquels sont confrontés les individus dans ce domaine.
    \item Traiter de l'épuisement professionnel : Les auteurs soulignent le problème de l'épuisement professionnel chez les ingénieurs en informatique, qui peut résulter de facteurs tels que le manque d'indépendance ou de contrôle au travail, la pression prolongée et la surcharge d'informations. Les recherches futures pourraient approfondir les stratégies de prévention et de gestion de l'épuisement professionnel dans les équipes de développement de logiciels.
    \item Intégrer les approches centrées sur l'humain : Les auteurs suggèrent que les approches centrées sur l'humain, telles que la méthodologie agile avec des équipes autonomes qui valorisent les individus et les interactions, peuvent être bénéfiques pour le développement de logiciels. Cependant, ils soulignent également la nécessité d'équilibrer les gains de productivité avec la durabilité des ingénieurs et d'aborder les effets secondaires potentiels tels que la charge cognitive et la pression constante sur la livraison d'un code fonctionnel.
    \item Développement de lignes directrices validées empiriquement : Les auteurs proposent le développement de lignes directrices et de meilleures pratiques validées empiriquement pour mesurer, améliorer et maintenir la durabilité du point de vue de l'ingénieur. En identifiant les facteurs qui ont un impact sur la durabilité au niveau individuel et leur interaction avec les pratiques de l'équipe et de l'organisation, la recherche future peut contribuer à permettre un développement logiciel de haute qualité tout en garantissant le bien-être de l'ingénieur.
    \item Intégrer les domaines de recherche connexes : Les auteurs suggèrent de tirer parti des contributions de domaines de recherche connexes tels que les aspects humains du génie logiciel, la motivation et la cognition. En considérant ces domaines dans le contexte du développement de logiciels, les chercheurs peuvent mieux comprendre comment améliorer la durabilité des ingénieurs et la qualité globale des logiciels.
\end{enumerate}
Dans l'ensemble, les auteurs soulignent l'importance de prendre en compte la dimension humaine de la durabilité dans l'ingénierie logicielle, d'aborder des questions telles que l'épuisement professionnel, d'incorporer des approches centrées sur l'humain, de développer des lignes directrices et d'intégrer des domaines de recherche connexes afin d'améliorer le bien-être et l'efficacité des ingénieurs en logiciel dans ce domaine.

\paragraph{Questions de recherche}
Les questions de recherche exposées dans l'article "Sustainable software engineering - have we neglected the software engineer's perspective" visent à comprendre les facteurs qui ont un impact sur la durabilité d'un ingénieur dans le développement de logiciels. Ces questions de recherche servent de point de départ à l'étude de la dimension individuelle (humaine) de la durabilité dans le contexte de l'ingénierie logicielle :

\begin{enumerate}
    \item Comment la durabilité d'un ingénieur est-elle perçue dans la recherche et la pratique de l'ingénierie logicielle ?
    \item Quels sont les facteurs qui ont un impact sur la durabilité d'un ingénieur ? 
        \item Quelles sont les sources de stress internes et externes pour les ingénieurs en logiciel ?
        \item Quel est l'impact d'un cycle de développement court sur la durabilité d'un ingénieur ?
        \item Quel est l'impact des réunions (par exemple, réunion quotidienne, planification, rétrospectives, etc.) sur le bien-être d'un ingénieur ?
    \item Quels sont les facteurs à adapter et comment, dans le cycle de vie du développement agile, pour assurer la durabilité de l'ingénieur ?
    \item Comment pouvons-nous mesurer l'efficacité de l'amélioration en ce qui concerne la durabilité de l'ingénieur?
\end{enumerate}
Ces questions de recherche visent à explorer divers aspects liés à la durabilité des ingénieurs logiciels, y compris leur perception dans la recherche et la pratique, les facteurs influençant leur durabilité, l'adaptation des facteurs dans le développement agile pour assurer la durabilité, et les méthodes pour mesurer l'efficacité des améliorations dans la durabilité de l'ingénieur.

\paragraph{Démarche adoptée}
L'approche adoptée dans l'article "Sustainable software engineering - have we neglected the software engineer's perspective" consiste à analyser des revues systématiques de la littérature et des études de cartographie afin d'identifier les lacunes dans la recherche liée à la dimension individuelle (humaine) de la durabilité dans l'ingénierie logicielle. L'étude vise à attirer l'attention sur l'aspect négligé de la durabilité du point de vue de l'ingénieur et à souligner la nécessité d'approfondir les recherches sur les facteurs ayant un impact sur la durabilité de l'ingénieur dans le développement de logiciels. En s'appuyant sur la littérature existante et des études secondaires, la recherche cherche à jeter les bases de futures recherches empiriques et du développement de lignes directrices et de meilleures pratiques pour mesurer, améliorer et maintenir la durabilité du point de vue de l'ingénieur.

\paragraph{Implémentation de la démarche}
L'approche décrite dans l'article "Sustainable software engineering - have we neglected the software engineer's perspective" est mise en œuvre à travers les étapes suivantes :
\begin{enumerate}
    \item Analyse des revues systématiques de la littérature : La recherche implique l'analyse des analyses documentaires systématiques existantes et des études de cartographie dans le domaine de la durabilité de l'ingénierie logicielle. En examinant et en synthétisant les résultats de ces études, les auteurs visent à identifier les lacunes et les limites de la recherche actuelle en ce qui concerne la dimension individuelle (humaine) de la durabilité dans l'ingénierie logicielle.
    \item Identification des lacunes de la recherche : Grâce à l'analyse des revues de la littérature et des études de cartographie, les auteurs identifient les domaines dans lesquels la perspective de l'ingénieur individuel sur la durabilité a été négligée ou sous-représentée. En identifiant ces lacunes, la recherche vise à mettre en évidence la nécessité d'une investigation et d'une exploration plus approfondies des facteurs ayant un impact sur la durabilité de l'ingénieur dans le développement de logiciels.
    \item Proposer des orientations de recherche futures : Sur la base de l'analyse de la littérature existante et de l'identification des lacunes de la recherche, les auteurs proposent des orientations et des pistes de recherche futures pour aborder la dimension individuelle (humaine) de la durabilité dans l'ingénierie logicielle. Cela inclut le développement de lignes directrices et de meilleures pratiques validées empiriquement pour mesurer, améliorer et maintenir la durabilité de l'ingénieur sur le terrain.
    \item Définition des objectifs de recherche : La recherche fixe des objectifs de recherche spécifiques, tels que l'analyse du processus de développement de logiciels du point de vue de la durabilité de l'ingénieur, l'identification de mesures d'amélioration et la compréhension des facteurs ayant un impact sur la durabilité de l'ingénieur à un niveau individuel. Ces objectifs servent de feuille de route pour la poursuite de la recherche empirique et des enquêtes sur le terrain.
\end{enumerate}
En mettant en œuvre ces étapes, la recherche vise à contribuer à une meilleure compréhension du point de vue de l'ingénieur sur la durabilité dans l'ingénierie logicielle et à fournir des idées et des recommandations pour améliorer le bien-être et l'efficacité des ingénieurs en logiciel dans l'industrie.

\paragraph{Les résultats}
Les résultats de l'article "Sustainable software engineering - have we neglected the software engineer's perspective" mettent en évidence les principales conclusions suivantes.

L'analyse des revues systématiques de la littérature et des études de cartographie révèle une lacune dans la recherche sur la dimension individuelle (humaine) de la durabilité dans l'ingénierie logicielle. Alors qu'il existe des recherches sur les dimensions environnementales et énergétiques de la durabilité, l'aspect humain a été relativement négligé.

La recherche actuelle en génie logiciel s'est principalement concentrée sur la réduction de l'empreinte énergétique des systèmes logiciels. Cet accent souligne la nécessité de réorienter l'attention vers le point de vue de l'ingénieur individuel sur la durabilité et d'aborder les facteurs ayant un impact sur son bien-être et ses besoins professionnels.

Les résultats suggèrent qu'il est à nouveau nécessaire de prêter attention au point de vue de l'ingénieur sur la durabilité dans le développement de logiciels. En prenant en compte des facteurs tels que l'épuisement professionnel, le stress et les défis liés au travail, les recherches futures peuvent contribuer à améliorer la durabilité des ingénieurs et à permettre un développement logiciel de haute qualité.

Les résultats soulignent l'importance d'incorporer des approches centrées sur l'humain dans le développement de logiciels afin d'équilibrer les gains de productivité avec la durabilité de l'ingénieur. En prenant en compte les facteurs qui ont un impact sur le bien-être et l'efficacité des ingénieurs logiciels, les chercheurs peuvent développer des lignes directrices et des bonnes pratiques pour assurer un développement durable des logiciels.
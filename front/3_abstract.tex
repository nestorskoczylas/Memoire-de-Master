%----------------------------------------------------------------------------------------
%	ABSTRACT PAGES
%----------------------------------------------------------------------------------------

% IMPORTANT NOTE: the abstract must always be written in two languages. If the report
% is written in Portuguese you have selected 'portuguese' as the language in the document class.
% Therefore, the portuguese version of the abstract must come first, so write it in the
% below area denoted by 'MAIN LANGUAGE ABSTRACT'. The english version follows in the
% 'SECOND LANGUAGE ABSTRACT' section.
% If the report is written in English, first will come the abstract in English
% ('MAIN LANGUAGE ABSTRACT') and then in Portuguese ('SECOND LANGUAGE ABSTRACT').

\begin{abstract}
%%%%%%%%%%%%%%%%%%%%%%%%%%%%%% MAIN LANGUAGE ABSTRACT %%%%%%%%%%%%%%%%%%%%%%%%%%%%%%%%%%

Le mémoire explore l'intégration de la durabilité dans le développement logiciel, soulignant l'importance cruciale d'adopter des pratiques écologiques, sociales et économiques pour garantir un avenir durable à l'industrie du logiciel.
Il s'appuie sur une approche interdisciplinaire combinant génie logiciel, durabilité environnementale et sciences sociales pour répondre à la question centrale : \textit{Comment incorporer les principes de développement durable et écologique dans le processus de développement de logiciels pour créer des applications respectueuses de l'environnement ?}
L'analyse approfondie des défis et des opportunités révèle que la durabilité ne peut être considérée comme une simple option, mais doit être intégrée de manière transversale et stratégique dans tous les aspects du développement logiciel. L'intégration de la durabilité dans le processus de développement logiciel offre des avantages significatifs en termes d'efficacité opérationnelle, de responsabilité sociale et environnementale.
Les pratiques d'ingénierie logicielle écologique, telles que l'optimisation de l'efficacité du code et la réduction de la consommation de ressources, ont le potentiel de réduire de manière significative l'impact environnemental de l'industrie du logiciel.
L'engagement en faveur de l'ingénierie logicielle durable nécessite une collaboration étroite entre les acteurs de l'industrie, les chercheurs et les décideurs pour développer des solutions innovantes et durables.
En intégrant la durabilité dans le processus de développement, les équipes de développement peuvent contribuer de manière significative à la création d'un avenir plus vert et plus responsable pour le secteur du logiciel.
La durabilité dans le développement logiciel est un impératif pour l'industrie du logiciel afin de répondre aux défis environnementaux et sociaux actuels.
L'engagement en faveur de la durabilité dans le développement logiciel est une démarche gagnant-gagnant qui peut conduire à des solutions plus performantes sur le plan commercial tout en réduisant l'impact environnemental de l'industrie du logiciel.
La recherche et l'innovation continue dans le domaine de l'ingénierie logicielle durable sont essentielles pour faire progresser les pratiques et les normes de l'industrie, contribuant ainsi à la construction d'un avenir plus durable et plus éthique pour la société dans son ensemble.

%----------------------------------------------------------------------------------------

\vspace*{10mm} 
\noindent
\textbf{\keywordslabel}: durabilité, développement logiciel, pratiques durables, impact environnemental, éco-conception logicielle, recherche empirique, rse, sdlc

%%%%%%%%%%%%%%%%%%%%%%%%% END OF THE MAIN LANGUAGE ABSTRACT %%%%%%%%%%%%%%%%%%%%%%%%%%%%%%
\end{abstract}
\begin{secondlangabstract}
%%%%%%%%%%%%%%%%%%%%%%%%%%%%%% SECOND LANGUAGE ABSTRACT %%%%%%%%%%%%%%%%%%%%%%%%%%%%%%%%%%

This thesis explores the integration of sustainability into software development, emphasizing the crucial importance of adopting ecological, social, and economic practices to ensure a sustainable future for the software industry.
The thesis draws on an interdisciplinary approach combining software engineering, environmental sustainability, and social sciences to answer the central question: How to incorporate sustainable and ecological development principles into the software development process to create environmentally friendly applications?
The in-depth analysis of challenges and opportunities reveals that sustainability cannot be considered a mere option but must be integrated transversally and strategically into all aspects of software development. Integrating sustainability into the software development process offers significant benefits in terms of operational efficiency, social responsibility, and environmental responsibility.
Green software engineering practices, such as optimizing code efficiency and reducing resource consumption, have the potential to significantly reduce the environmental impact of the software industry.
Commitment to sustainable software engineering requires close collaboration between industry players, researchers, and decision-makers to develop innovative and sustainable solutions.
By integrating sustainability into the development process, development teams can significantly contribute to creating a greener and more responsible future for the software industry.
Sustainability in software development is an imperative for the software industry to address current environmental and social challenges.
Commitment to sustainability in software development is a win-win approach that can lead to more commercially successful solutions while reducing the environmental impact of the software industry.
Continuous research and innovation in the field of sustainable software engineering are essential to advance industry practices and standards, contributing to building a more sustainable and ethical future for society as a whole.


%----------------------------------------------------------------------------------------

\vspace*{10mm} 
\noindent
\textbf{\keywordslabel}: sustainability, software development, sustainable practices, environmental impact, software eco-design, empirical research, csr, sdlc

%%%%%%%%%%%%%%%%%%%%%%%%%% END OF THE SECOND LANGUAGE ABSTRACT %%%%%%%%%%%%%%%%%%%%%%%%%%%%%
\end{secondlangabstract}


\section{Livrable 2 : Plan détaillé}

\subsection{Articles}

\subsubsection{Article principal}

1. Green Software Engineering with Agile Methods | \href{https://ieeexplore.ieee.org/document/6606425}{https://ieeexplore.ieee.org/document/6606425} | ICSE International Conference on Software Engineering | Rank A*

2. The GREENSOFT Model: A reference model for green and sustainable software and its engineering | \href{https://www.sciencedirect.com/science/article/abs/pii/S2210537911000473}{https://www.sciencedirect.com/science/article/abs/pii/S2210537911000473} | Sustainable Computing: Informatics and Systems | Rank Q1

3. An Empirical Study of Practitioners' Perspectives on Green Software Engineering | \href{https://dl.acm.org/doi/abs/10.1145/2884781.2884810}{https://dl.acm.org/doi/abs/10.1145/2884781.2884810} | 2016 IEEE/ACM 38th IEEE International Conference on Software Engineering | Rank A*

4. Safety, Security, Now Sustainability: The Nonfunctional Requirement for the 21st Century | \href{https://ieeexplore.ieee.org/document/6728940}{https://ieeexplore.ieee.org/document/6728940} | IEEE Software | Rank Q1

\subsubsection{Article secondaire}

1. Integrating Sustainability Metrics into Project and Portfolio Performance Assessment in Agile Software Development: A Data-Driven Scoring Model | \href{https://www.mdpi.com/2071-1050/15/17/13139}{https://www.mdpi.com/2071-1050/15/17/13139} | Journal : Sustainability | Rank Q1

2. Green and Sustainable Software Engineering - a Systematic Mapping Study | \href{https://dl.acm.org/doi/abs/10.1145/3275245.3275258}{https://dl.acm.org/doi/abs/10.1145/3275245.3275258} | SBQS '18: Proceedings of the XVII Brazilian Symposium on Software Quality | Rank B

3. Sustainability is Stratified: Toward a Better Theory of Sustainable Software Engineering | \href{https://ieeexplore.ieee.org/abstract/document/10172842}{https://ieeexplore.ieee.org/abstract/document/10172842} | 2023 IEEE/ACM 45th International Conference on Software Engineering (ICSE) | Rank A*

4. The Green Software Measurement Structure Based on Sustainability Perspective | \href{https://ieeexplore.ieee.org/document/9611108}{https://ieeexplore.ieee.org/document/9611108} | 2021 International Conference on Electrical Engineering and Informatics (ICEEI) | Rank C

5. Sustainable software engineering - have we neglected the software engineer's perspective? | \href{https://ieeexplore.ieee.org/document/9679832}{https://ieeexplore.ieee.org/document/9679832} | 2021 36th IEEE/ACM International Conference on Automated Software Engineering Workshops (ASEW) | Rank A*

6. The Impact of Green Feedback on Users’ Software Usage | \href{https://ieeexplore.ieee.org/document/9953563}{https://ieeexplore.ieee.org/document/9953563} | IEEE Transactions on Sustainable Computing | Rank Q1

7. Application of the Sustainability Awareness Framework in Agile Software Development | \href{https://ieeexplore.ieee.org/document/10260996}{https://ieeexplore.ieee.org/document/10260996} | 2023 IEEE 31st International Requirements Engineering Conference (RE) | Rank A*

8. From Sustainability in Requirements Engineering to a Sustainability-Aware Scrum Framework | \href{https://ieeexplore.ieee.org/document/9604667}{https://ieeexplore.ieee.org/document/9604667} | 2021 IEEE 29th International Requirements Engineering Conference (RE) | Rank A*

\section{Problématique}

Comment incorporer les principes de développement durable et écologique dans le processus de développement de logiciels en vue de créer des applications plus respectueuses de l'environnement ?

\section{Plan}

\subsection{Introduction}
\begin{itemize}
    \item Contexte : L'augmentation de la consommation d'énergie des technologies de l'information et de la communication (TIC) et le besoin d'intégrer des principes de durabilités dans le développement de logiciels.
    \item \textit{Problématique} : Comment incorporer les principes de développement durable et écologique dans le processus de développement de logiciels en vue de créer des applications plus respectueuses de l'environnement ?
    \item Objectifs : Proposer des modèles de mesure intégrant les aspects de durabilité, examiner les approches pour une pratique durable du génie logiciel, et analyser les impacts des pratiques individuelles des ingénieurs logiciels sur la durabilité. (articles principaux 1, 2, 3, 4 ; articles secondaires 1, 2, 3, 4, 5, 6, 7, 8)
\end{itemize}

\subsection{Mesure de la Durabilité dans le Développement Logiciel} (articles principaux 1, 2, 3; articles secondaires 1, 2, 3)
\begin{itemize}
    \item Modèle de Mesure de la Performance et de Durabilité : \\
    Comment concevoir un modèle de mesure intégrant efficacement la durabilité dans l'évaluation des performances des projets logiciels ?
    \item État de l'Art des Approches Durables dans le Génie Logiciel : \\
    Quelles sont les approches existantes pour la pratique durable du génie logiciel, et comment elles sont classifiées et évaluées ?
    \item Théorie de la Durabilité en Génie Logiciel : \\
    Comment les diverses dimensions de la durabilité sont-elles conceptualisées dans la littérature académique en génie logiciel ?
\end{itemize}

\subsection{Pratiques Individuelles des Ingénieurs Logiciels et Durabilité (articles principaux 3, 4 ; articles secondaires 4, 5, 6, 7)}
\begin{itemize}
    \item Influence des Pratiques Individuelles sur la Durabilité : \\
    Comment les pratiques individuelles des ingénieurs logiciels influent-elles sur la durabilité des logiciels ?

Article 3:
"Only when meeting performance goals becomes egregious in terms of power, then we negotiate a compromise that balances performance and power consumption."
"I care about memory usage, CPU usage, like I understand those. [...] I don’t have the same intuition about energy."
"Energy concerns influence how practitioners write new code."

Article 4:
"Software engineers can considerably improve civilization’s sustainability by taking into account not just the first-order impacts of software systems, but also their second- and third-order impacts."
"It isn’t civilization’s intention to harm the Earth, but the collective sum of our individual actions, which often favor local convenience over global responsibility, added to the effects of the societal structures we’ve created lead to negative consequences for our environment."
"If our civilization is to transition to sustainability, many sectors of society will need to rethink their modes of operation."
"The challenge of incorporating this conflict into software engineering is an issue of requirements prioritization, which is usually handled by negotiation between system stakeholders." 
"Perhaps the solution requires an explicit stakeholder for environmental sustainability."

Article 1.4:
"Green software measurement aims to provide and offer software practitioners and stakeholders a mechanism to measure the green compliance of software products, which will help them ensure their software's naturalness in their organizations." 
"We can accomplish green software product by applying the complete three elements of sustainability as mentioned in the previous section of this paper: social, economic, and environmental." 
"The required measurements of sustainable features define for evaluating the target. It comprises measures that influence green products, as mentioned in [33]. If the measurements are good in sustainability dimensions, the delivered product would also have a high level of green." 
"Green software elements are decomposed into measurements and sub measurements. Then, the sub measurement is broken down to a further level of decomposition that associate with direct assessment metrics." 
"Sustainable software is related to its influence on the economy, society, people, and the environment. It is the consequence of minor upgrades, delivery, and use of software and positively impacts sustainability."

Article 1.5:
"The capacity of the human working memory and the amount of cognitive load it can process (i.e., cognitive bandwidth) are closely related." 
"Overloading a human’s 'limited' working memory inhibits his learning ability and problem-solving skills." 
"Cognitive load is considered a waste in SE." 
"A lack of independence or control at work and prolonged pressure increases the risks of burnout in software engineers." 
"Developing software under such circumstances not only affects the mental health of the engineers but also compromises the quality of the produced software."

Article 1.6:
"In contrast, users who stated to change behavior when doing less important tasks, did not translate into power reductions." 
"We find that green feedback helps in raising awareness about software energy, and on the willingness of users to apply energy-efficient changes." 
"Participants also miss the tools and the knowledge of what to do to change their behavior, even if they want to." 
"Software power consumption in mobile devices vary depending on the device itself."

Article 1.7:
"The quality of the sustainability workshops, their outcomes, and the mappings between the identified sustainability effects and the backlog items are influenced by the knowledge, experience, and understanding of the first author who conducted." 
"The results also depend on the experience of the first author who led the workshops, and also on the perception of sustainability by the other workshop participants."  

    \item Mesures pour un Génie Logiciel Vert : \\
    Comment mesurer et améliorer la durabilité des logiciels du point de vue des pratiques individuelles des ingénieurs ?

Article 3:
"The responses for Statement S8 show that 80\% of respondents consider energy concerns when they write new code Sometimes, Often or Almost Always."
"Practitioners believe that they do not have accurate intuitions about the energy usage of their code."
"The finding that 'energy concerns influence how practitioners write new code' suggests that new programming languages or language features could help developers during the development of energy-efficient applications."

Article 4:
"To ensure that such regulations and standards have the desired effects, it might be necessary to trigger institutional change so that environmental regulations consider more than first-order effects."
"For both standards and quality assurance, we need to define a set of metrics for the different dimensions of sustainability by relying on the respective sets of metrics available."
"Assessment Techniques: For assessment techniques to address quality assurance, we propose evaluating and adapting LCA to software engineering and making use of environmental impact assessment in software engineering."
"The IEEE 1680 family of standards for environmental assessment of IT is already taking steps in that direction."
"Aiming for environmental sustainability standards to support constraint specification, the next step is to extend software engineering standards."

Article 1.4:
"Green software and its products are crucial in solving the problems associated with the long-term use of software, especially from a sustainability point of view." 
"Green software is accomplished by adopting minimum waste generation in the development and operation process." 
"Green software engineering is a fundamental software engineering activity in the 21st century. It has been constant challenges that prompted explorations on green information technology in the software industry." 
"Many works in green software development do not emphasize green measurement that focuses on sustainability. The primary target for green software is to produce sustainable software products with a less negative effect on the environment." 
"Green software measurement aims to provide and offer software practitioners and stakeholders a mechanism to measure the green compliance of software products, which will help them ensure their software's naturalness in their organizations."

Article 1.5:
"Future research should leverage contributions from related research areas like human aspects in software engineering (e.g., topics like cognition and motivation)." 
"There is a need for identifying factors that impact sustainability at an individual level and their interplay with the team and organization level practices, policies, and decisions." 
"The overall ambition is to develop empirically validated guidelines and best practices to measure, improve and maintain sustainability from an engineer’s perspective." 
"To enable high-quality software development, it is essential to realize the engineer’s personal, professional needs and maintain their well-being."

Article 1.6:
"Live green feedback helps in raising awareness of energy consumption." 
"Specific green metrics, such as power consumption, electricity price, or CO2 emissions, do not seem to have an effect on end users." 
"Participants either exaggerated or minimized the energy costs, and nearly half of them did not understand what the metrics mean." 
"Users need to know whether switching to a different program will lead to energy reductions."

Article 1.7:
"Furthermore, we also noticed the need for a tool that supports recording new sustainability effects exposed by specific backlog items." 
"For practical and regular usage, this procedure is quite inefficient as it is difficult to track any changes and to visualize results." 
"The specified time of the workshops limited the space for a more detailed explanation of sustainability in software engineering." 
"To mitigate this threat, the participants were briefed on sustainability and the dimensions of sustainability before the workshops." 
"We conducted sustainability workshops based on SusAF and mapped identified sustainability effects to the product backlog items of both projects." 

    \item Contribution au Développement Durable via le Comportement Utilisateur : \\
    Comment le comportement des utilisateurs peut-il contribuer à la durabilité des logiciels, et quel est l'impact de la sensibilisation et des retours d'information sur ce comportement ?

Article 3:
"Energy usage should be a shared responsibility."
"I could learn how to improve energy usage by reading documentation."
"Good energy usage should be the responsibility of applications."

Article 4:
"If sustainability policies and standards are put in place and software engineers prioritize them in the systems they develop, future technology may significantly contribute indirectly to influencing the behavior of users who interact with those systems in some ways while directly contributing to saving the planet."
"To support the transition to sustainability, environmental sustainability must be explicitly considered as a nonfunctional requirement in the software engineering process."
"Software engineers can contribute to sustainability by designing software systems that minimize indirect impacts and by educating users about the impacts of their actions."
"Software engineers must accept the challenge of integrating sustainability into the systems we build."
"The ubiquity of software offers a unique opportunity: a shared point of intervention across a wide range of intertwined and impactful industries."

Article 1.4:
"The software can minimise power consumption by being more energy-efficient (i.e., by becoming greener); utilizing lesser power or adopting more sustainable and supported procedures will diminish the environmental effects of software used by governments, organizations, and people." 
"Green software products and processes can also be related to cost-effectiveness with auto-coding and efficient codes provided in maintenance activity. With this effort, software defects can be predicted and reduce defect occurrence during software execution." 
"Sustainability in software is often associated with long-life span software to operate in its development till operation and removal. Previous studies have revealed that sustainability depends on the quality level of the software." 
"Sustainable software is related to its influence on the economy, society, people, and the environment. It is the consequence of minor upgrades, delivery, and use of software and positively impacts sustainability." 
"Sustainability also relates to the ability to improve the failure of a system function in the future. For example, the system cannot interact with the user due to changes in the environment and cannot support system architecture."

Article 1.6:
"Green feedback helps in raising awareness, but users lack the knowledge and tools to apply software behavioral changes." 
"Green feedback tools must be minimal and seamlessly integrated to avoid being distracted." 
"Participants wanted to change their behavior but had a few obstacles to do so." 
"Our main conclusion is that green feedback helps in raising awareness, but users lack the knowledge and tools to properly adopt lasting and energy-effective behavioral changes."

Article 1.7:
"Although we received positive feedback from the practitioners concerning the continuous discussion of sustainability effects of software systems during the development process, some participants also mentioned several challenges." 
"First, regarding sustainability issues in software development is not seen as best practice yet and will only be taken into account if explicitly requested by the clients." 
"Second, since there is a fear that it is very time-consuming but also cost-intensive, long-term benefits have to be evident." 
"Third, open questions are who or which role should be responsible for sustainability, who has appropriate knowledge of sustainability concerns, or how can such knowledge be obtained." 
"Even more, reviewing backlog items towards sustainability will be almost impossible for external persons that are not regularly involved in the development process."

\end{itemize}

\subsection{Pratiques Durables au Niveau Global (articles principaux 1, 2, 4 ; articles secondaires 1, 3, 5, 7, 8)}
    \begin{itemize}
    \item Intégration de la Durabilité dans les Projets Logiciels : \\
    Comment mettre en œuvre des pratiques durables au niveau d'un projet logiciel ? Quels sont les modèles et les outils disponibles pour évaluer et améliorer la durabilité à l'échelle du projet ?

Article 1:
"In order to implement the presented procedure models, the actors in software development projects need to be supported by tools: Besides educational material regarding sustainability issues, specialized software, and spread sheets, these are tools, well-known in the field of software development (e.g. Continuous Integration)." 
"Using Continuous Integration enables developer teams to minimize their integration effort and their feedback on current software quality. We wanted to use these assets on rating the energy efficiency of software. This is why we developed a model to integrate performance and energy measurement into Continuous Integration."

Article 2:
"Albertao et al. [26,27] interpret common software quality properties and associated metrics on the background of SD. They classify the quality properties into development, usage, and process related properties."
"Arndt et al. [28] discuss implications of evolving releases of a widely used text processor and relate these to Green IT and SD. As a solution to cope with sustainability issues during software design and development, they propose the so called “Grand Management Information Design”."
"The second part of the GREENSOFT Model is called Sustainability Criteria and Metrics. It covers common metrics and criteria for the measurement of software quality [43] and it allows a classification of criteria and metrics for evaluating a software product’s sustainability."
"Our model has the ability to represent three categories of sustainability criteria and metrics for software products: Common Quality Criteria and Metrics, Directly Related Criteria and Metrics, and Indirectly Related Criteria and Metrics."
"Besides reflecting the proposed life cycle of a software product, there are further methods that support software architects, designers, and developers in producing green and sustainable software applications."

Article 4:
"Sustainability in software engineering not only encompasses energy efficiency and green IT, but must also consider the second- and third-order impacts of software systems." 
"For assessment techniques to address quality assurance, we propose evaluating and adapting LCA to software engineering and making use of environmental impact assessment in software engineering."

Article 1.1:
"The proposed model enables the measurement of the sustainability of code, thereby fostering a more comprehensive understanding of the relationship between project delivery performance and sustainable practices."

Article 1.3:
"The goal of this model is to explain how sustainability relates to software development based on existing qualitative research and to provide a more nuanced view of the stratified and multisystemic nature of sustainability according to our observations from the meta-synthesis." 
"The model separates the software process from the software product to illustrate that sustainability is a property of both and while the nature of the software development process affects the resulting software system, a sustainable process does not guarantee a sustainable product or vice versa." 
"The model depicts four dimensions or pillars of sustainability—environmental, social, economic and technical. All four dimensions apply to software products."

Article 1.5:
"The third outcome is the development of empirically validated practices and guidelines for processes, policies, practices to ensure engineer’s sustainability." 
"The resulting taxonomy of factors and guidelines will influence the decision-making models and methods to take into consideration the concepts of sustainability." 
"The factors would also serve as a foundation for future research to find interrelations between them, categorize them and devise ways to control them."

Article 1.7:
"Our future work will support the development of a Sustainability-Aware Scrum Framework, which could provide guidelines on how to consider sustainability issues in ASD, based on the original Scrum framework." 
"In this experience paper, we present the results of two case studies from ongoing agile development projects where the Sustainability Awareness Framework was applied." 
"We conducted sustainability workshops based on SusAF and mapped identified sustainability effects to the product backlog items of both projects."

Article 1.8:
"Duboc et al. [11], [15] present a sustainability awareness framework (SuSAF) that includes a set of instructions and questions that can be used by requirements engineers to guide discussions on sustainability with the stakeholders."
"Paech, Moreira, Araujo, and Kaiser introduce an iterative process with two checklists. The first checklist can be used to identify needs of each sustainability dimension for a system, e.g. little pollution and little waste in the environmental dimension or high customer satisfaction and little cost in the economical dimension."
"Penzenstadler et al. describe the application of the framework in an educational context."
"It is remarkable that none of these approaches directly target the application within an agile software development setting, although this does not mean that they are not applicable in such a setting."
    
    \item Pratiques Durables au Niveau de l'équipe de Développement : \\
    Comment une équipe de développement peut-elle adopter des pratiques durables au quotidien ? Quelles sont les initiatives qui peuvent être mises en place pour encourager la durabilité au sein de l'équipe ?

Article 1:
"According to the definitions and the life cycle, future impacts on sustainable development, which are expected to arise from post development phases, have to be considered during software development. This means that actors have to anticipate and estimate first-, second-, and third-order impacts." 
"To organize and establish a behavior in a software development process that is aware of sustainability issues, software developing organizations should implement a continuous improvement cycle that focuses on relevant effects and impacts." 
"The enhancements proposed in the following section are based on the definitions of Green and Sustainable Software Engineering given above. The life cycle serves as a tool that provides hints on which effects and impacts actors should consider in software development processes."

Article 2:
"Depending on the area of application, it may be possible to use approximations instead of accurate calculations, which may require less processing."
"Organizations that develop green and sustainable software should commit themselves to environmental and social responsibility, expressed, e.g. in environmental and social responsibility statements, their commitment to international labor standards."
"It should be mentioned, that this procedure model sets only an organizational framework that helps managing sustainability of a software product."
"The model component Procedure Models makes it possible to classify procedure models that cover acquisition and development of software, maintenance of IT systems, and user support."
"On the one hand, there are tools that automatically calculate software metrics from source code or compiled artifacts."
"Producing ecologically sound, resource and energy efficient software is also an issue nowadays."

Article 4:
"Thorough research requires evaluating various methods through case studies in a variety of application domains to develop guidance on the most appropriate, adapted methods in the context of supporting environmental sustainability as it pertains to software systems." 
"To ensure that such regulations and standards have the desired effects, it might be necessary to trigger institutional change so that environmental regulations consider more than first-order effects." 
"We’ve described a number of areas in software engineering that can borrow from safety and security to address sustainability in effective ways; each of these areas requires extended further research." 

Article 1.1:
"The proposed key performance indicators (KPIs) not only facilitate a better understanding of the frequency and velocity of value delivery but also emphasize the importance of sustainable practices in software development." 
"The insights gained from this study underscore the pivotal role of a harmonized delivery and sustainability metrics system in enhancing the sustainability and efficiency of software development undertakings." 
"This research endeavors to lay the groundwork in this direction, advocating the principles of continuous innovation in project and portfolio management." 

Article 1.3:
"Since countless interventions could potentially improve the sustainability of software processes and products, many researchers can contribute to this area in parallel." 
"Interventions may be social, technical, or sociotechnical." 
"Interventions may focus on software products or software processes." 
"More collaboration between researchers and professionals is needed both to develop and to evaluate sustainability interventions that are useful and usable in real-world software development settings." 

Article 1.5:
"Agile methodology emerged as a human-centric approach with autonomous teams that value individuals and interactions over processes and tools." 
"Developing software under such circumstances not only affects the mental health of the engineers but also compromises the quality of the produced software." 
"To enable high-quality software development, it is essential to realize the engineer’s personal, professional needs and maintain their well-being."

Article 1.7:
"Our hypothesis is that continuously analyzing sustainability effects of new backlog items during the development process will provide new information to the product owner and the stakeholders for decision making." 
"This suggests that a sustainability analysis of backlog items can support the early identification of potential sustainability effects in Scrum." 
"We demonstrated that SusAF can be applied in agile contexts as well, and that backlog items can be successfully linked to sustainability effects derived from the software system’s product visions."

Article 1.8:
"A potential software tool that tracks the sustainability impacts of software requirements (defined as user stories), is related to RQ3, RQ4 and RQ5, because it can support the whole Scrum team during the backlog refinement in analysing and prioritising the requirements."
--
RQ3, RQ4, and RQ5 refer to specific research questions outlined in the document regarding sustainability-aware Scrum framework development:

RQ3: "Which artefacts of Scrum (definition of ready, acceptance criteria, etc.) can be adapted to foster the sustainability of a software system and how can they be used in a sustainability-aware Scrum framework?"
RQ4: "Which roles within a Scrum framework can support the product owner to consider sustainability aspects?"
RQ5: "Which events of the Scrum workflow (sprint planning, backlog refinement, etc.) can be used to address sustainability issues and how can this be accomplished?"
--
"Adding sustainability aspects to Scrum artefacts (user stories, definitions of ready, acceptance criteria, etc.), has a positive effect on the identification, analysis, documentation, validation and management of sustainability impacts of the software system."
"Addressing sustainability impacts in different events of the Scrum workflow (backlog refinement, product review, retrospective, etc.), has a positive effect on the identification, analysis, documentation, validation and management of sustainability impacts of the software system."
    
    \item Durabilité et Stratégie d'Entreprise : \\
    Comment la stratégie de RSE d'une entreprise peut-elle être alignée avec les objectifs de durabilité dans le développement logiciel ? Quels sont les avantages et les défis liés à l'incorporation de la durabilité dans la culture d'entreprise ?

Article 1:
"From our point of view, design for sustainable development is a principle that should be present in every software project and product, even in general purpose software where the application domain does not directly target on improving another process’ or product’s impacts or effects on sustainable development." 
"In order to develop green and sustainable software, one has to know how to define green and sustainable software and moreover the software process."

Article 2:
"Organizations that develop green and sustainable software should commit themselves to environmental and social responsibility, expressed, e.g. in environmental and social responsibility statements, their commitment to international labor standards."
"These commitments should also cover environmental and social standards throughout the entire supply chain of all products and services, which are necessary to produce, advertise, distribute, and dispose/recycle the software product or parts of it."
"For standard software products, energy efficiency and hardware obsolescence criteria can be either used as technical requirements or as award criteria."
"Purchasing software or purchasing the appropriate hardware for a software product is also of high relevance for home users."
"It is clear, that home users, as well as purchasers of micro and small enterprises, require information on sustainability issues that can be obtained easily."
"This may be accomplished by printing the according information on product boxes or product sheets."

Article 4:
"With regard to a software system’s impact on sustainability, we distinguish three orders of magnitude. First-order impacts are direct effects of a software system on its environment." 
"The reason is that the requirements usually reflect incomplete or wrong assumptions about the operations or context. Extensive investigation into specification and analysis of requirements for safety-critical systems began in the 1990s, especially in the area of formal methods." 
"More elaborate theoretical frameworks for defining sustainability exist, all of which rely on systems thinking to some extent." 

Article 1.1:
"The proposed model serves as a beacon for software organizations worldwide, promoting enhanced delivery efficiency while bolstering the sustainability ethos of the software development lifecycle." 
"The research not only presents this model but also has practical implications. The findings underscore the model’s unparalleled efficacy, providing a pragmatic solution for the simultaneous assessment of project and portfolio performance with a dual emphasis on delivery and sustainability."
"The emphasis on sustainability principles is set to amplify. This research endeavors to lay the groundwork in this direction, advocating the principles of continuous innovation in project and portfolio management." 

Article 1.3:
"Meanwhile, our analysis discovered zero controlled experiments; indeed, the dominant research method was non-empirical (e.g. position papers)." 
"In conclusion, SE research should (1) propose and rigorously assess more sustainability-improving interventions, and (2) develop more sophisticated instruments that account for the different meanings of sustainability at different strata and the diverse social, technical, and sociotechnical systems affected." 
"More research on SSE processes is needed if professionals wish to achieve sustainability throughout software development and not just in the usage and maintenance of software products." 
"Software professionals must also be encouraged to view sustainability not just as an ecological concept or a technical requirement, but also as a concern of human and economic stakeholders." 

Article 1.5:
"The resulting taxonomy of factors and guidelines will influence the decision-making models and methods to take into consideration the concepts of sustainability for example, decisions regarding team structure, organization structure, policies and prevailing software development practices." 
"The current software engineering literature lacks the individual (human) dimension of sustainability." 
"As software has become an essential part of our lives, its ever-increasing usage has also increased demands for energy and resource consumption."

Article 1.7:
"Although we received positive feedback from the practitioners concerning the continuous discussion of sustainability effects of software systems during the development process, some participants also mentioned several challenges." 
"First, regarding sustainability issues in software development is not seen as best practice yet and will only be taken into account if explicitly requested by the clients." 
"There might even be the need for legal requirements to perform sustainability assessments before and/or during developing new software systems."

\end{itemize}

\subsection{Conclusion}
\begin{itemize}
    \item Réponse à la problématique : \\
    Synthèse des résultats pour répondre à la question de l'intégration réussie de la durabilité dans le développement logiciel, en prenant en compte les aspects de performance, de pratiques durables, et d'impacts individuels.
    \item Limites de l'étude : \\
    Identification des limites des modèles proposés et des résultats de l'étude empirique.
    \item Perspectives : \\
    Proposition de pistes de recherche futures pour améliorer la durabilité dans le développement logiciel, en mettant l'accent sur les pratiques individuelles et l'interaction avec les utilisateurs.
\end{itemize}

Liens utiles :
- Schéma représentant la durabilité au sein du génie logiciel : \href{https://luiscruz.github.io/course_sustainableSE/2022/}

- Responsible software: A research agenda to help enterprises become more sustainable
\href{https://dspace.library.uu.nl/handle/1874/347113}

What is the problem with this article? You need to write more or less one page.

What are the possible avenues pointed out by the authors?

What are the research questions?

What is the approach adopted? Do not explain how it was implemented.

How is the approach being implemented?

What are the results?